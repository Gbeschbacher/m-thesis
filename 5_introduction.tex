\section{Introduction}

The number of computers is increasing steadily. People build new devices to assist in various ways. Devices in smart buildings for example should work together to synergise and do not contradict in being useful in the respective fields. Furthermore automatic cars have to communicate with other cars and the traffic infrastructure to safely drive and reach their destination in a fast and efficient way. Nowadays such computers get more ubiquitous and thus have to become more intelligent in various ways. It gets more common to not manually construct intelligence rather than letting machines learn themselves the optimal approach. This usually takes enormous amounts of time to process and even more if in an environment in which every other computer is in a state of evolution too. Meaning that the learning machine cannot rely on a stationary domain but rather take changes into consideration and use past events as references, interpret and use them to predict future actions.

\bigskip

\ac{ai} is defined as the ability of machines to act and think similar to humans \iacite[29]{millington.2009}. Usually the behaviour of an \ac{ai} is abstracted by the game designer and then implemented. This way they are solving the specific problem in the intended way. Most often those \acp{ai} are inflexible and tend to fail to perform in different domains.