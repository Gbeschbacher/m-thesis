\section{Introduction}
%add more about generating data and data overall
There has never been such a quantity of data available in history and the global amount of information has never grown so fast. Every day 2.5 Exabytes of new data is generated and by 2020 an annual growth of 35 Zettabytes is predicted \iacite{IBM.2013}. Most scientific experiments and simulations of different types like chemistry, medicine or physics generate more and more data. ``The Large Hadron Collider'' of CERN in Genf produced 13 petabytes of data in 2010 \iacite{Brumfield.2011}. Economy, especially merchandise trade generates data in large measures. Each and every one of us generates data every day. Partly, this happens knowingly with activities in social media \textendash\ Facebook alone produces 500 Terabytes every day \iacite{Tam.2012} \textendash\ but we constantly produce data with various sensors in our smartphones.

%add more about information overload and turning information into knowledge
Processing and analysis of data garners increased economic and scientific interest. Searching for useful and usable information is commonly known as ``Data Mining''.

Despite recent technical advantages of search engines and content provider services the information overload problem still remains a crucial issue for many application fields. Finding the right piece of information in huge information spaces is a tedious and time consuming task. Visualisations have shown to be an effective way to deal with the overload issue providing the possibility to display and explore a huge set of data points at the same time. However, creating useful visual representations of data typically requires domain knowledge \iacite{Nazemi.2013}.

% Assume a dataset including retail information.
For example, assuming a visualisation about e-commerce data exists, there is often the problem that it is not intuitive and easy to understand. If a shop wants to find out where to open their next warehouse in order to provide a better delivery service, it is mandatory to find out where most of the orders come from. For this type of task, multiple types of visualisations exist. Using thematic cartography is one way to address the given task. Displaying all orders in a map results in constructing first knowledge about the geographic distribution, which could already be sufficient. However, there are many different thematic maps in the field of cartography and some may not be as appropriate as others. Giving users control to change the visual appearance may be sufficient to satisfy their needs. Additionally, animating the transition between appropriate visual appearances could result in a better readability of each visualisation.
% Considering the aforementioned task, every thematic map has a certain use-case and animating the transition between maps could support the understanding of the primal use-case of the map.

Thus the importance of the field of visualisation grows with the amount of generated data. Visualisation utilises the human capability of understanding pictures better than numbers. It allows the discovery of knowledge in data that would not be possible with other means. Maps are most efficient in enabling perspicuity of complex situations. They can be seen as an interface between a user and the data. Thus, enabling them to answer location-related questions \iacite{Gartner2013}.

\subsection{Problem Statement and Contribution}
Today, maps can be created and used by any individual from virtually any location on earth and for any purpose. This new mapmaking paradigm often leads to misuse of different types of maps, even though cartographic principles remain unchanged \iacite{Gartner2013}.
% TODO: mention data-overload issue again in combination with spatial sensors and the missing evaluation through visualisation of these.
Thus, this thesis aims at the creation of a visualisation tool that animates the transition between different thematic maps. The system should help users to identify an appropriate visual representation according to their task. The main requirement for the design and implementation of the tool are:
\begin{itemize}
\item The dataset used to show on a map should contain more than 5000 items
\item There should at least be the choice between three different kinds of thematic maps
\item Changing the visual appearance needs to be animated
\item Modularity of the system is mandatory in order to extend it with different types of transitions and visualisation
\end{itemize}
Currently available tools do not meet these requirements to a satisfactory degree and therefore, this thesis focuses on the following question:
which tasks can be supported by particle aggregation in a thematic map and how can an animated aggregation be realized?

\subsection{Structure of this Document}
In the remainder of this chapter, the history of visualisations is discussed first, followed by a basic definition of visualisation with different varieties of this research field. Afterwards, basic principles of visual representations are explained. Next, exploratory data anlysis is explained, followed by visual analytics with a focus on visual channels and their usage.
The examples and applications shown in this thesis are mainly from the domain of thematic cartography. To brief the reader, the start of chapter \ref{s:methods} gives an introduction to this domain. Furthermore, this chapter includes interaction principles for visualisations with a state-of-the-art application analysis of related work. Chapter \ref{s:results} shows the implementation of the practical application. The core of this thesis is chapter \ref{s:discussion}. It discusses the implementation and the methods used addressing the hypothesis. The thesis is concluded in chapter \ref{s:outlook}.

\subsection{Visualisation Background}
\section{Visualization}
Visualization is not a new phenomenon. It has been used in maps, scientific drawings and data plots for over a thousand years. Examples of this are the map of China and the famous map of Napoleon's invasion of Russia which are explained later in detail. Most of the concepts learned in devising these images carry over in a straight forward manner to computer visualization \iacite{Tufte:1990}.

This first part of this chapter provides an overview important milestones in the history of visualization. The history is followed by a definition of visualization in literature and a definition of visualization for this thesis. This yields to the basis to be able to define different types of visualizations like \ac{InfoVis}, \ac{SciVis} and \ac{GeoVis} (which are all explained in detail in chapter \ref{s:definitions-types} on page \pageref{s:definitions-types}). The definitions are followed by an essential discussion about the basic principles of data visualizations and differnet types of visualizations.

The second part of tis chapter uses the knowledge of the first part to describe exploratory data analysis in conjunction with visualizations. This is also done with some pratical applications and examples which are also used to segue into the next chapter which deals with \ac{GeoVis}. It is used to show different types and use cases of \ac{GeoVis}. Therefore some related works are mentioned and their problems are mentioned.

\subsection{History}
\label{s:history}
This entire chapter is based on the work of \citeauthor{Friendly.2001}, who made an interactive timeline of milestones of the history of visualizations. Even though \citeauthor{Friendly.2001} provides a huge amount of historical visualizations and informations, it is not possible to discuss and mention everything because it would go beyond the scope and therefore only the important visualizations for this thesis are used \iacite{Friendly.2001}.

\begin{quote}
    The earliest seeds of visualization arose in geometric diagrams, in tables of the positions of stars and other celestial bodies, and in the making of maps to aid in navigation and exploration \iacite{Friendly.2001}.
\end{quote}

One of the oldest maps of the roman empire is dated back to the 4\textsuperscript{th} century A.D. In 1508, Konrad Peutinger came in the way of such a map and it was named after him at a later date: Tabula Peutingeriana or Peutinger map. Before Peutinger got the map it was copied multiple times from the world map of Agrippina (from the 1\textsuperscript{st} century B.C.). Figure \ref{fig:peutinger} on page \pageref{fig:peutinger} shows the peutinger map which had a length of 680cm and was drawn on a long book-roll.

\begin{figure}[!htb]
\centering
\includegraphics[width=0.8\textwidth,keepaspectratio]{images/history/peutinger.png}
\caption[
    Peutinger map, Urldate: 07.2016 \newline
\small\texttt{\url{https://web.archive.org/web/20080129123649/http://www.kargi.de/Geschichte/Peutinger/Peutinger.bmp}}
]{Peutinger map}
\label{fig:peutinger}
\end{figure}

Figure \ref{fig:peutinger-rome} on page \pageref{fig:peutinger-rome} shows Rome as the central point with 12 different ways leaving it. The map was not meant to be accurate and true to scale. Its main goal was to give travelers outline where they need to go to reach their destination. All distances in the map are graphically distorted but the correct distances are mentioned next to the roads as numbers. Thus the map's only accuracy consists of the structure (left and right).

\begin{figure}[!htb]
\centering
\includegraphics[width=0.5\textwidth,keepaspectratio]{images/history/peutinger_rom.jpg}
\caption[
    Peutinger map with Rome as central point, Urldate: 07.2016 \newline
\small\texttt{\url{https://web.archive.org/web/20060106224928/http://www.kargi.de/Geschichte/Peutinger/peutinger_rom.jpg}}
]{Peutinger map with Rome as central point}
\label{fig:peutinger-rome}
\end{figure}

A greek astronomer called Ptolemy was one of the most influential greek astronomers and geographers of his time. He was one of the first to propound the geocentric theory of the solar system. His maps projected the earth spherical where latitude combined with longitude characterized positions. Alongside the first use of longitude in maps he also was the first one specifying terrestrial locations by celestial observations. Figure \ref{fig:ptolemy} on page \pageref{fig:ptolemy} shows a reconstructed map from Ptolemy's geography in the 15\textsuperscript{th} century, indicating "Sinae" at the far right, beyond the oversized island of "Taprobane" and the "Aurea Chersonesus".

\begin{figure}[!htb]
\centering
\includegraphics[width=0.8\textwidth,keepaspectratio]{images/history/ptolemy-map.jpg}
\caption[
    Reconstructed Ptolemy's map, Urldate: 07.2016 \newline
\small\texttt{\url{https://upload.wikimedia.org/wikipedia/commons/2/23/PtolemyWorldMap.jpg}}
]{Reconstructed Ptolemy's map}
\label{fig:ptolemy}
\end{figure}

% TODO: ueberleitung von maps auf andere visualisierungen
The earliest known attempt to show changing values graphically was around the 10\textsuperscript{th} century. Figure \ref{fig:planetary-movement} on page \pageref{fig:planetary-movement} tries to show planetary movement like the positions of the sun, moon and planets throughout the year.

\begin{figure}[!htb]
\centering
\includegraphics[width=0.4\textwidth,keepaspectratio]{images/history/planetary-movement.jpg}
\caption[
    Planetary movements, Urldate: 07.2016 \newline
\small\texttt{\url{http://www.fi.uu.nl/wiskrant/artikelen/hist_grafieken/begin/images/planeten.gif}}
]{Planetary movements}
\label{fig:planetary-movement}
\end{figure}

1350 was the year in which the first bar chart of its sort appeared and was named proto-bar graph. Oresme, a french bishop and philosopher, proposed the use of a bar graph for plotting a variable magnitude whose value depends on another. Figure \ref{fig:oresme-proto} on page \pageref{fig:oresme-proto} shows such a graph in detail and figure \ref{fig:oresme-page} on page \pageref{fig:oresme-page} shows a page written in Latin containing different types of concepts which all implicitly have the idea of a coordinate system. The graphical use of bars also reveal the implication of the use of mathematical aggregation.

\begin{figure}[!htb]
\centering
\includegraphics[height=4cm,keepaspectratio]{images/history/oresme-proto.jpg}
\caption[
    Oresme proto bar graph, Urldate: 07.2016 \newline
\small\texttt{\url{http://datavis.ca/milestones//admin/uploads/images/icons/oresmekl.gif}}
]{Oresme proto bar graph}
\label{fig:oresme-proto}
\end{figure}

\begin{figure}[!htb]
\centering
\includegraphics[height=10cm,keepaspectratio]{images/history/oresme-page.jpg}
\caption[
    Oresmes concept page written in Latin, Urldate: 07.2016 \newline
\small\texttt{\url{http://datavis.ca/milestones//admin/uploads/images/oresme6.gif}}
]{Oresmes concept page written in Latin}
\label{fig:oresme-page}
\end{figure}

25 years later, 1375, Abraham and Jehuda Cresques made the first catalan atlas. It consists of six images also called portulan charts which show the position of coasts and ports very accurately. All portulan charts together cover the regions from the Atlantic Ocean to China. The informations to make the atlas was gathered from sailors who used Mallorca, the hometown of Abraham and Jehuda Cresques, as a junction. Figure \ref{fig:catalan-atlas} on page \pageref{fig:catalan-atlas} shows one portulan chart displaying Europe, North Africa and the Near East. The atlas was the first of its kind having undistorted distances between geographical regions.

\begin{figure}[!htb]
\centering
\includegraphics[width=0.8\textwidth,keepaspectratio]{images/history/catalan-atlas.jpg}
\caption[
    One out of six portulan charts of the catalan atlas, Urldate: 07.2016 \newline
\small\texttt{\url{http://datavis.ca/milestones//admin/uploads/images/CatalanE.jpg}}
]{One out of six portulan charts of the catalan atlas}
\label{fig:catalan-atlas}
\end{figure}

1569 Gerard Mercator invented a new map projection which bears his name. The main task for this projection was to map a world on to a cylinder in such a way that all lines of latitude have the same length as the equator. This should help seamen to lay out a course easily because a nagivator needed a map where a line of constant bearing would cross all meridians at the same angle. On the one hand the projection provides very accurate directions and shapes of regions, but on the other hand the size of regions are distorted incresingly to the north and south pole. Figure \ref{fig:mercator} on page \pageref{fig:mercator} shows a map of the world with a mercator projection.

\begin{figure}[!htb]
\centering
\includegraphics[width=0.8\textwidth,keepaspectratio]{images/history/mercator.png}
\caption[
    A map of the world with a mercator projection, Urldate: 07.2016 \newline
\small\texttt{\url{https://upload.wikimedia.org/wikipedia/commons/b/b2/Mercator_1569.png}}
]{A map of the world with a mercator projection}
\label{fig:mercator}
\end{figure}

One year later, 1570, the first modern atlas named Teatrum Orbis Terrarum ("Theatre of the World"), or Ortelius-Atlas appeared. It was a book consisting of a collection of uniform map sheets and sustaining text bounds. The collection of map sheets actually included not a single map made by Ortelius. At first it bundled maps from 53 other cartographers with the source. However Ortelius brought consistency to all those maps. He dropped the maps all in the same style and size and arranged them logically by continent, region and state. Nevertheless the naming and location coordinates were not normalized. Figure \ref{fig:ortelius} on page \pageref{fig:ortelius} shows a world map in Teatrum Orbis Terrarum. Ortelius also regularly revised and expanded the atlas making it the first economically successful one.

\begin{figure}[!htb]
\centering
\includegraphics[width=0.8\textwidth,keepaspectratio]{images/history/ortelius.jpeg}
\caption[
    A map of the world in the Ortelius atlas, Urldate: 07.2016 \newline
\small\texttt{\url{https://upload.wikimedia.org/wikipedia/commons/6/6f/OrteliusWorldMap.jpeg}}
]{A map of the world in the Ortelius atlas}
\label{fig:ortelius}
\end{figure}

1626 the idea of "small multiples" got developed. It was used to show a series of images in a coherent display. The origin of the idea is based in displaying the changes in sunspots over time in a visual representation as figure \ref{fig:small-multiples} on page \pageref{fig:small-multiples} shows.

\begin{figure}[!htb]
\centering
\includegraphics[height=5cm,keepaspectratio]{images/history/small-multiples.png}
\caption[
    Sunspot plate from Scheiner's "Tres Epistolae" using small multiples, Urldate: 07.2016 \newline
\small\texttt{\url{http://cnx.rice.edu/content/m11970/latest/tres_epistolae.gif}}
]{Sunspot plate from Scheiner's "Tres Epistolae" using small multiples}
\label{fig:small-multiples}
\end{figure}

20 years later the first visual representation of statistical data was made. Langren showed the variations in determination of longitude between Toledo and Rome (see figure \ref{fig:langren} on page \pageref{fig:langren}). The left-most location is the starting point in this visualization. The distance to other locations is shown as a number line.

\begin{figure}[!htb]
\centering
\includegraphics[width=0.8\textwidth,keepaspectratio]{images/history/langren.jpg}
\caption[
    First data graph showing variations in determination of longitude between Toledo and Rome, Urldate: 07.2016 \newline
\small\texttt{\url{http://datavis.ca/milestones//admin/uploads/images/tufte/langren.jpg}}
]{First data graph showing distance between different locations.}
\label{fig:langren}
\end{figure}

In the 18\textsuperscript{th} century, cartographers began to try to show more than one channel on a map (the term "channel" is explained in detail in chapter \ref{s:basics} on page \pageref{s:basics}). Up to now the only channel which got displayed on a map was just geographical position. As a result, new graphic forms such as contours lines, or also called isolines, got invented and thematic mapping of physical quantities took root. Towards the end of this century, the first attempts at thematic mapping of geologic, economic, and medical data are visible.

The first half of the 19\textsuperscript{th} century witnessed a tremendous growth in statistical graphics and thematic mapping because of the fertilization provided by previous innovations of design and technique. All modern forms of data visualizations now known as bar and pie charts, histograms, line graphs and time-series plots, contour plots and so forth, have been invented. In the field of cartography, a shift from single geographical maps to comprehensive atlases, depicting data on a wide variety of topics (economic, social, medical, physical, etc.) happened. As an example, figure \ref{fig:first-choropleth} on page \pageref{fig:first-choropleth} features the first choropleth map (for more information about choropleth maps see chapter \ref{s:choropleth} on page \pageref{s:choropleth}). It shows the distribution and intensity of illiteracy in France.

\begin{figure}[!htb]
\centering
\includegraphics[height=5cm,keepaspectratio]{images/history/dupin.jpg}
\caption[
    First choropleth map showing the distribution and intensity of illiteracy in France, Urldate: 07.2016 \newline
\small\texttt{\url{http://datavis.ca/milestones//admin/uploads/images/dupin.gif}}
]{First choropleth map showing the distribution and intensity of illiteracy in France}
\label{fig:first-choropleth}
\end{figure}

This map was followed shortly by comparative choropleth thematic maps, showing crimes against persons and crimes against property in relation to level of instruction by departments in France (see figure \ref{fig:second-choropleth} on page \pageref{fig:second-choropleth}.).

\begin{figure}[!htb]
\centering
\includegraphics[height=5cm,keepaspectratio]{images/history/second-choropleth.jpg}
\caption[
    The first comparative choropleth thematic maps, showing crimes against persons and crimes against property in relation to level of instruction by departments in France, Urldate: 07.2016 \newline
\small\texttt{\url{http://datavis.ca/milestones//admin/uploads/images/guerry/guerry-balbi-600s.jpg}}
]{The first comparative choropleth thematic maps, showing crimes against persons and crimes against property in relation to level of instruction by departments in France}
\label{fig:second-choropleth}
\end{figure}

1830 the first dot map of population appeared in France. Each dot represents 10.000 people in the department (see figure \ref{fig:first-dotmap} on page \pageref{fig:first-dotmap}.).

\begin{figure}[!htb]
\centering
\includegraphics[width=0.8\textwidth,keepaspectratio]{images/history/montizon-dotmap2.jpeg}
\caption[
    The first dot map of population by department, 1 dot = 10.000 people, Urldate: 07.2016 \newline
\small\texttt{\url{http://gallica.bnf.fr/ark:/12148/btv1b8492261j/f1.highres}}
]{The first dot map of population by department, 1 dot = 10.000 people}
\label{fig:first-dotmap}
\end{figure}

The second half of the 19\textsuperscript{th} century is also known as "the golden age of data graphics". All the conditions for tremendous growth of visualizations were given once again: throughout Europe official state statistical offices were established.
Statistical theory (initiated by Gauss and Laplace) provided the means to make sense of large bodies of data.
This half of the century included the first mixture of a map with a diagram: figure \ref{fig:first-mixture} on page \pageref{fig:first-mixture} shows a map using pie charts to represent the cattle sent from all around France for consumption in Paris.

\begin{figure}[!htb]
\centering
\includegraphics[width=0.4\textwidth,keepaspectratio]{images/history/minard.png}
\caption[
    A map using pie charts to represent the cattle sent from all around France for consumption in Paris., Urldate: 07.2016 \newline
\small\texttt{\url{https://upload.wikimedia.org/wikipedia/commons/1/1c/Minard-carte-viande-1858.png}}
]{A map using pie charts to represent the cattle sent from all around France for consumption in Paris.}
\label{fig:first-mixture}
\end{figure}

Another well known example of graphical representation from that century is the so called "cholera map" (see figure \ref{fig:cholera-map} on page \pageref{fig:cholera-map}.). Here, a dot map is used to display epidemiological data. This map could be used for knowledge creation in terms of the discovery of the source of a cholera epidemic. It showed that a high number of deaths were occuring near a water pump.

\begin{figure}[!htb]
\centering
\includegraphics[width=0.8\textwidth,keepaspectratio]{images/history/cholera2.png}
\caption[
    Dot map showing the clusters of cholera cases in the London epidemic of 1854., Urldate: 07.2016 \newline
\small\texttt{\url{http://datavis.ca/milestones//admin/uploads/images/tufte/snow.gif}}
]{Dot map showing the clusters of cholera cases in the London epidemic of 1854.}
\label{fig:cholera-map}
\end{figure}

In 1861, a new channel appeared in map based visualizations. It represents a weather map with glyphs (as a new channel) where each glyph embodies air pressure and barometric changes by means. Figure \ref{fig:weather-map} on page \pageref{fig:weather-map} shows the mentioned weather map.

\begin{figure}[!htb]
\centering
\includegraphics[width=0.4\textwidth,keepaspectratio]{images/history/weather.jpg}
\caption[
    A chart showing area of similar air pressure and barometric changes by means of glyphs displayed on a map., Urldate: 07.2016 \newline
\small\texttt{\url{http://datavis.ca/milestones//admin/uploads/images/galton-weather-charts2.gif}}
]{A chart showing area of similar air pressure and barometric changes by means of glyphs displayed on a map.}
\label{fig:weather-map}
\end{figure}

8 years later, Napoleon's march on Moscow (also called "the best graph ever produced") was visualized by Minard and is shown in figure \ref{fig:minard2} on page \pageref{fig:minard2}. Overall this figurative map can be described as the "Map of the successive losses in men of the French Army in the Russian campaign 1812-1813". The reason why this graph is sometimes referred to as the best graph ever produced is, because it contains multiple visual channels without influencing the comprehensibility of the graph. The following list describes some of the values encoded in the graph:
\begin{itemize}
\item The amount of men alive is encoded by the thickness of the colored lines in a rate of one millimeter for 10.000 men. The exact amount of men alive is also written beside the zones.
\item The color of the lines indicates the moving direction of the troops. Brown designates men moving into Russia and black are those on retreat.
\item The temperature is plotted at different points along the retreat at the bottom.
\end{itemize}
All in all, figure \ref{fig:minard2} on page \pageref{fig:minard2} encodes six different types of data in two dimensions:
\begin{enumerate*}
\item the number of Napoleon's troops,
\item the distance traveled
\item temperature
\item latitude and longitude
\item direction of travel
\item location relative to specific dates.
\end{enumerate*}
As of today, such a visual representation is known as a Sankey diagram which is a specific type of a flow diagram.

\begin{figure}[!htb]
\centering
\includegraphics[width=0.8\textwidth,keepaspectratio]{images/history/minard2.png}
\caption[
    Charles Minard's map of Napoleon's disastrous Russian campaign of 1812., Urldate: 07.2016 \newline
\small\texttt{\url{https://upload.wikimedia.org/wikipedia/commons/2/29/Minard.png}}
]{Charles Minard's map of Napoleon's disastrous Russian campaign of 1812.}
\label{fig:minard2}
\end{figure}

If the literature refers to the 19\textsuperscript{th} century as "the golden age", they also often refer to the first half of the 20\textsuperscript{th} century as the "modern dark ages" of visualization. Only a few graphical innovations arised because in this period statistical graphics became to be common practice. Nonetheless, visual representations were used to provide new insights, discoveries and theories in this period. This was probably the first time graphical methods were used in different research fields such as astronomy, physics, and other sciences to justify hypotheses. Another research, which had a lot of impact in the research field of data visualization, was about comparisons of the efficacy of various graphic forms. In the mid 1960s, three developments made a significant impact on data visualizations:
\begin{enumerate}
\item \ac{EDA} was first mentioned by John W. Tukey in a paper called "The Future of Data Analysis". He demanded a strict distinction between data analysis as a legitimate branch of statistics and mathematical statistics.
\item The book called "Sémiologie graphique" (Semiology of graphics) was written and published by Jacques Bertin. To some
\begin{quote}
"[\ldots] this appeared to do for graphics what Mendeleev had done for the organization of the chemical elements, that is, to organize the visual and perceptual elements of graphics according to the features and relations in data \iacite{Friendly.2001}."
\end{quote}
\label{crossref:bertain}
\item Computer processing of data arised, which offered more flexibility and possibilities in constructing graphic forms by programs.
\end{enumerate}

By the end of the 20\textsuperscript{th} century new research fields emerged and significant intersections and collaborations began: computer science research combined with the developments in data analysis provided new paradigms, languages and helpful packages in order to create statistical and data graphics. Thus a tremendous growth in new visualization methods and techniques appeared anew.
Another important innovation in that period were new paradigms  concerning direct manipulation of data (linking, brushing, selection, focus, etc. (for a detailed explanation see chapter \ref{s:basics} on page \pageref{s:basics}.)).

\subsection{Definition}
\label{s:definition}
\cbstart
The main goal of this Section is giving an overview of the definitions in literature of the term \textit{visualisation}. The first definitions appeared are taken from the beginnings of the research field of visualisation. The latter ones bear a reference to the technology available nowadays.
\cbend
Visualisation is first mentioned but not defined in 1953 in cartographic literature, in an article by University of Chicago geographer Allen K. Philbrick. \citeauthor{mccormick:1987} took up the term and defined it as follows:
\begin{quote}
 ``Visualisation is a method of computing. It transforms the symbolic into the geometric, enabling researchers to observe their simulations and computations. visualisation offers a method for seeing the unseen. It enriches the process of scientific discovery and fosters profound and unexpected insights. In many fields, it is already revolutionizing the way scientists do science.

 Visualisation embraces both image understanding and image synthesis. That is, visualisation is a tool for both for interpreting image data fed into a computer, and for generating images from complex multi-dimensional data sets. It studies those mechanisms in humans and computers which allow them in concert to preceive, use and communicate visual information \iacite{mccormick:1987}.''
\end{quote}

In 1987, new developments in the field of computer science prompted the National Science Foundation to redefine the term. The report of the redefinition placed visualisation at the convergence of computer graphics, image processing, computer vision, computer-aided design, signal processing, and user interface studies \iacite{mccormick:1987}. \citeauthor{Phillips2010} discuss three elemental issues existing today in the research field of visualisation:

\begin{enumerate}
\item to settle on a definition for the term \textit{visualisation}
\item to clarify the underlying presumptions and
\item to decide how to document both short-term and long-term effectiveness.
\end{enumerate}

\begin{quote}
The status of terms, often used interchangeably, such as \textit{visualisation}, \textit{visual representation}, \textit{visual media}, \textit{media literacy}, \textit{visual communication skills}, \textit{visual literacy}, \textit{illustrations}, and \textit{media illustrations}, is yet to be clarified. Furthermore, the routine confusion between pictures or visual images and reality is a fundamental and persistent problem \iacite{Phillips2010}.
\end{quote}

Because of the wide reach of the term, \citeauthor{Phillips2010} use a series of five steps in order to tell how \textit{visualisation} is defined in literature. The steps can be summarized as follows:
\begin{enumerate}
\item A search of all relevant sources, the identification of vocabulary and the mapping of the citations.
\item Classify the types of research into explanatory, exploratory, descriptive studies and ``other''.
\item Analyse and evaluate the assertions made in step two.
\item Organization of the reviews through comparisons of the literature, in order to identify areas of difference and similarity.
\item Mapping the collected information on several categories.
\end{enumerate}
The evaluation of this method used a total of 247 articles, ranging from the year 1936 to 2009. Out of those articles, 56\% were empirical studies and 44\% discussion articles. \citeauthor{Phillips2010} found out, that the attempt to define \textit{visualisation} in literature is not possible. The term is frequently substituted with terms like \textit{visual aid}, \textit{image}, \textit{visual literacy} etc.

To clarify the term with a relation to the progress made the last years, two well known online dictionaries are used. One possible generic definition could be the one from Oxford Dictionaries\footnote{See \href{https://www.oxforddictionaries.com/definition/english/visualisation}{https://www.oxforddictionaries.com/definition/english/visualisation}}:

\begin{quote}
\begin{enumerate}
\item ``The representation of an object, situation, or set of information as a chart or other image.''
\item ``The formation of a mental image of something.''
\end{enumerate}
\end{quote}

Even though the noun \textit{visualisation} has a very close related verb, \citeauthor{Phillips2010} noted the important distinction between those. The noun ``[\ldots] directs the attention to the product, the object, the 'what' of visualisation, the visual images. The verb of visualisation, on the other hand, makes one attend to the process, the activity, the skill, the 'how' of visualizing.''\iacite{Phillips2010}.

Another possible generic definition for \textit{visualisation} is taken from the Merriam Webster Online Dictionary\footnote{See \href{http://www.merriam-webster.com/dictionary/visualization}{http://www.merriam-webster.com/dictionary/visualization}}. It is close to the one taken from the Oxford Dictionaries but has one significant distinction: ``The act or process of interperting in visual terms or of putting into visible form''. Compared to the statements of \citeauthor{Phillips2010}, this dictionary does not distinguish the process and the act of \textit{visualisation}.

However, combining the already mentioned definitions of terms with the research results from \citeauthor{Phillips2010}, a three-fold distinction of definitions is visible:
\begin{enumerate}
\item Physical objects serving as visualisations (e.g. geometrical illustrations). These objects are viewed and interpreted by a person for the purpose of understanding something.
\item Mental objects pictured in the mind (e.g. mental imagery). These are imaginative constructions of some possible visual experiences.
\item Cognitive processing in which visualisations are interpreted, either physical or mental (e.g. cognitive manipulation of visual representations by the mind). This often refers to an act of deriving meaning from a physical or mental object.
\end{enumerate}

\cbstart
\citeauthor{Munzner2014} takes the available technology into account and therefore provides a more advanced definition. She combines the term \textit{visualisations} with computer-based systems yielding new definition possibilities to the whole term. Nonetheless, she does not revoke the older definitions and defines the term on the basis of them.
Using computer-based systems help people to carry out tasks more effectively.
\textit{Visualisations} created by systems cannot replace a human with computational decision-making methods, they are needed when there is a need to augment human capabilities. Furthermore, besides taking the process of creation into account, \textit{visualisations} also need to consider interaction with visual representations \iacite{Munzner2014}.

According to \citeauthor{Phillips2010} the distinction between physical and mental visualisation objects are obvious. This statement is also verified by the given definition of \citeauthor{mccormick:1987} where this exact distinction is already made. However the distinction between the visualisation itself and the thinking involved in interpreting it is also important \iacite{Phillips2010}. \citeauthor{Munzner2014} extends the term of \textit{visualisation} with the combination of computer-based systems and the thereby existing possibilities of interaction yielding a much broader term with three more limitations:
\begin{enumerate*}[label={(\arabic*)}]
\item the limitations of computers,
\item of humans, and
\item of displays \iacite{Munzner2014}.
\end{enumerate*}

\cbend

\subsection{Varieties of visualizations}
\label{s:definitions-types}
In order to fully understand this section, the definition of the term \textit{visualisation} needs to be clear (see Chapter \ref{s:definition} on page \pageref{s:definition} for more information).

So far the term of visualisation has been defined. Formerly, the research field of visualisation was often split in two major parts: the field of \ac{InfoVis} and \ac{SciVis}. The following list briefly describes each field. However, the distinction between those fields becomes more and more blurred. This is due to the fact that the data used in visualisations are not necessarily of a single nature, e.g. geo-spatial data only. With today's possibilities in data acquisition and preprocessing, there are no limitations in combining data from different natures.

\begin{description}

\item[\acl{InfoVis}] \hfill \\
\citeauthor{Friendly.2001} refered to \ac{InfoVis} to the broadest term that could be used to group all the developments mentioned in Section \ref{s:history}. From a very basic point of view, almost everything, if sufficiently organised, contains information in some kind. Therefore the term could be used for the earliest attempts to scratch information on rocks and to the earliest use of diagrams in history \iacite{Friendly.2001}.

\citeauthor{Ferreira2003} had a more contemporary definition of \ac{InfoVis}. They said the graphical models may represent abstract concepts and relationships that do not necessarily have a counterpart in the physical world. An example of the mentioned relationship would be an information describing user accesses to pages of an internet portal. It is typical for each data unit to describe multiple related attributes (usually more than four) that are not of some kind of spatial or temporal nature. Although spatial and temporal attributes may occur, the data existed in an abstract data space \iacite{Ferreira2003}.
\newpage
\item[\acl{SciVis}] \hfill \\
A close related, yet distinct field, called \ac{SciVis}, was primarily concerned with the visualisation of 3D phenomena, where the emphasis was on realistic renderings of volumes, surfaces, etc. 3D Phenomena for \ac{SciVis} could be of architectural, meteorological, medical nature and so forth \iacite{Friendly.2001}.

\citeauthor{Ferreira2003} also had a similar definition for \ac{SciVis}. They defined the term as the graphical models which are typically constructed from measured or simulated data representing objects or concepts associated with phenomena from the physical world. As such, the derived visualisations represented objects that exist in a 1D, 2D or 3D object space. The presence of spatial and temporal dimensions could be included in the data and thus are determinants in deriving visual representations from it \iacite{Ferreira2003}.
\end{description}

In the early 1980s, a french graphic theorist named Jacques Bertin set a milestone in the area of scientific research, as already mentioned in Chapter \ref{s:history} on page \pageref{crossref:bertain}. Based on his work, the Semiology of Graphics, the research field called \ac{GeoVis} arose. His work shows a strong focus in the research of the potential for the use of dynamic visual displays as prompts for scientific insight and of the methods through which dynamic visual displays might leverage perceptual and cognitive processes to facilitate scientific thinking \iacite{maceachren:2004}.

The broad term \ac{GeoVis} has the same problems as the other one concerning a specific definition: the term is used in many different research fields. Thus, multiple definitions suitable for the particular case exist in literature. This thesis gives an overview of some definitions, which also apply to the use of the term in this thesis.
The following definition according to the 2001 research agenda of \ac{ICA}, Commission on visualisation and Virtual Environments, is most widely accepted:
\begin{quote}
``Geovisualisation integrates approaches from visualisation in scientific computing, cartography, image analysis, information visualisation, exploratory data analysis and geographic information systems to provide theory, methods, and tools for visual exploration, analysis, synthesis, and presentation of geospatial data \iacite{Longley2005}.''
\end{quote}

\citeauthor{Noellenburg2007} mentions other definitions which have a more human-centred view and describe geovisualisation as
\begin{quote}
``[\ldots] the creation and use of visual representations to facilitate thinking, understanding, and knowledge construction about geospatial data'', or as ``the use of visual geospatial to display and to explore data and through that exploration to generate hypotheses, develop problem solutions and construct knowledge \iacite{Noellenburg2007}.''
\end{quote}

\newpage
Even though the definitions are slightly constrasting, they share one thought: \ac{GeoVis} is a multidisciplinary task and since it is a human being using those visualisations to explore data and construct knowledge, \ac{GeoVis} takes the user needs into account above everything else.

\subsection{Basic Principles}
\label{s:basics}
\citeauthor{Munzner2014} answers a lot of questions concerning the usage of visualizations in her book called "Visualization Analysis and Design". This chapter will discuss the most important answers because they are a necessity in understanding questions like when, why and how to visualize.

From today's perspective with the ability to use artificial intelligence and machine learning, the question if we still need a human in order to create visualizations is very valid. This questions as a very simple answer: if well-defined questions to ask about data exist in advance, it is possible to answer these purely with computational techniques from fields such as statistics and machine learning. If a solution to a problem has been fully automatized and has been deemed to be acceptable, there is no need for human judgement anymore and thus no need to design a visualization tool. An example for a practical application would be the domain of stock market trading. A tool in that field can be fully automatized currently there are many deployed systems for high-frequency trading. Those systems make decisions about buying and selling stocks when certain market conditions hold \iacite{Munzner2014}. Some other reasons where computers beat humans are mentioned below:
\begin{itemize}
\item Scale: Drawing a dataset of hundreds of thousands of items by hand is infeasible and would take a lot of time.
\item Efficiency: Once a problem has been solved with a computer it can be reused  indefinitely for different datasets and scenarios.
\item Quality: Precise data-driven rendering.
\end{itemize}
However many analysis problems are very poorly specified. Either the approach to the problem is unknown, or it is not obvious which questions the data could or should answer. In such a case, \citeauthor{Munzner2014} says, that the best path forward is an analysis process with a human in the loop. Even though there are a lot of use cases for a human in the loop, this paper only discusses one because it deals with \ac{EDA} (see chapter \ref{s:eda} on page\pageref{s:eda} for more information.), which is essential for this thesis. Exploratory analysis in scientific discovery is a common case of the need of a human. A long-term visualization tool which could be developed is a tool with the goal of speeding up and improving user's ability to generate and check hypothesis \iacite{Munzner2014}.

% Maybe use ability matrix here

Another important question answered by \citeauthor{Munzner2014} is: why depend on vision? As chapter \ref{s:definition} on page \pageref{s:definition} already mentions, a part of visualization is based on exploiting the human visual system as a means of communication. A significant amount of visual information processing occurs in parallel through high-bandwidth channels to our brain. One example \citeauthor{Munzner2014} mentions is visual popout: one red item in a sea of gray ones is immediately noticed.

Furthermore a single static view can show only one aspect of a dataset. This fact is already part of the answer of the next question: why use interactivity? Even though there are combinations of simple datasets and tasks where only a single visual encoding and therefore a single static view is needed, it does not apply for large complex datasets where interactivity allows to change displays and supports many possible queries. Interactivity could be used to investigate multiple levels of detail at once, ranging from a very specific detailed view of a small part to a very high-level summarization \iacite{Munzner2014}.

The main goal of a design of a visualization is to satisfy rather than optimize. One of the many possible good solutions to a problem is much harder to find than one of the even larger number of bad ones. Yet the validation of satisfaction is very difficult because there are so many questions considering wheter a visualization tool has met the design goals \iacite{Munzner2014}.

Furthermore it is needed to keep at least three different kinds of limitations in mind when designing or analyzing visualizatoins:
\begin{enumerate}
\item computational capacity,
\item human perceptual and cognitive capacity and
\item display capacity.
\end{enumerate}

All three limitations can be subsumized with the term scalability.

One of the most important questions \citeauthor{Munzner2014} answer is called: why analyze? In order to answer the question they feature an analysis framework that helps to think about design choices for visualizations systematically. Figure \ref{fig:an-framework} on page \pageref{fig:an-framework} shows the high-level framework they provide for analyzing a visualization's use according to three questions: what data does a user see, why does the user intend to use a visualization tool and how are the visual encoding and interaction idioms constructed in terms of design choices \iacite{Munzner2014}.

\begin{figure}[ht]
\centering
\includegraphics[width=0.3\textwidth,keepaspectratio]{images/basics/analysis-framework.png}
\caption[
    Three-part analysis framework for a visualization instance: why is the task being performed, what data is shown in the views, and how is the vis idiom constructed in terms of design choices \iacite{Munzner2014}.
]{Three-part analysis framework for a visualization instance: why is the task being performed, what data is shown in the views, and how is the vis idiom constructed in terms of design choices.}
\label{fig:an-framework}
\end{figure}

\subsection{Exploratory data analysis}
\label{s:eda}
% This topic could be taken to subsume the two main focii: statistical graphics, and thematic cartography.

% Both of these are concerned with the visual representation of quantitative and categorical data, but driven by different representational goals. Cartographic visualization is primarily concerned with representation constrained to a spatial domain; statistical graphics applies to any domain in which graphical methods are employed in the service of statistical analysis. There is a lot of overlap, but more importantly, they share common historical themes of intellectual, scientific, and technological development.

% In addition, cartography and statistical graphics share the common goals of visual representation for exploration and discovery. These range from the simple mapping of locations (land mass, rivers, terrain), to spatial distributions of geographic characteristics (species, disease, ecosystems), to the wide variety of graphic methods used to portray patterns, trends, and indications.



\subsection{Geovisualization}
% In modern geovisualization software, such data are represented using both traditional cartographic techniques based on the use of colours, textures, symbols, and diagrams; and using computer-enabled techniques such as map animation and interactive 3D views. Moreover, maps are used in combination with nongeographic visualization techniques such as scatterplots or parallel coordinates. The use of multiple interactively linked views providing different perspectives into the data has become a kind of standard in geovisualization. However, a number of problems have yet to be solved, such as the scalability of geovisualization tools and their usability.

% Introduction

% Related Work

%     Visualizations

%         History

%         Definition

%         Basic principles

%         Exploratory data analysis
%             - theoretical statistics
%             - exploratory data analysis with visualizations

%         Definitions of differnt types of visualizations
%             - information visualization
%             - scientific visualization
%             - geovisualization

%         Geovisualization
%             - partical applications
%             - examples and use-cases

%             - aggregation in geovisualization
%             - related work

%     Transitions in Visualizations
%         - examples and use-cases
%         - related work

%         - Transitions in Geovisualizations
%             - related work and concepts

%     Particle-based related work

%     Summary of problems and missing transitions in related work

% Implementation

% Results

% Discussion & Outlook


\subsection{Collaboration Statement}
\label{s:collaboration-statement}
Aside from the supervisors of this thesis, \textbf{Lukas Wanko} contributed some parts of the work described in this thesis. He contributed on a conceptual level and to the implementation of the practical application. He focused on implementing different animated transition layouts between non thematic visualisations. However, this thesis will not include and discuss any part of this research. For more information, please see his masterthesis \iacite{Wanko2016}.