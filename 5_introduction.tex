\section{Introduction}

\bigskip

\ac{ai} is defined as the ability of machines to act and think similar to humans \iacite[29]{millington.2009}. Usually the behaviour of an \ac{ai} is abstracted by the game designer and then implemented. This way they are solving the specific problem in the intended way. Most often those \acp{ai} are inflexible and tend to fail to perform in different domains.

\section{Methods}

\section{Results}

\section{Discussion}

\section{Outlook}



\section{Related Work}
The first part of this chapter provides an overview of the related work in the field of
Information Visualization. First general methods, which are important for visualizations of heterogeneous data in this area of application, are described. This rather high level discussion is followed by a more specialized view on map-based visualizations and transitions in visualizations. The combination of these two emerging research areas leads to the application of map-based visualizations with transitions between them. Finally the current state in the domain of map-based visualization is summarized and potential improvements are identified.

\subsection{Geovisualization methods}
Visualization as a term is fist mentioned 1953 in the cartographic literature, in an article by University of Chicago geographer Allen K. Philbrick. 1987 new developments in the field of computer science prompted the National Science Foundation to redefine the term. The report of the redifinition placed visualization at the convergence of computer graphics, image processing, computer vision, computer-aided design, signal processing, and user interface studies \iacite{mccormick:1987}. Visualization now also emphasizes the knowledge creation and hypothesis generation aspects of \ac{scivis}.

In the early 1980s, a french graphic theorist named Jacques Bertin set a milestone in the area of scientific research. Based on his work in this \ac{geovis} developed as a field of research. His work shows a strong focus in the research of the potential for the use of dynamic visual displays as prompts for scientific insight and of the methods through which dynamic visual displays might leverage perceptual cognitive processes to facilitate scientific thinking \iacite{maceachren:2004}.

As already mentioned, \ac{geovis} is closely related to the fields of \ac{scivis} and also \ac{infovis} and emphasizes knowledge construction over knowledge storage or information transmission \iacite{maceachren:1997}. However \ac{infovis} needs to be strictly differentiated from \ac{scivis}. \ac{infovis} deals with abstract data like for example movies in a movie database whereas \ac{scivis} operates on real-world data with spatial character.

\ac{geovis} also contributes significantly to other related fields such as \ac{scivis} and also \ac{infovis}. Owing to its roots in cartography, \ac{geovis} contributes to these other fields by way of the map metaphor, which "[\ldots] has been widely used to visualize non-geographic information in the domains of information visualization and domain knowledge visualization" \iacite{Jiang2005}. It emphasizes knowledge construction over knowledge storage or information transmission \iacite{maceachren:1997}. However \ac{infovis} needs to be strictly differentiated from \ac{scivis}. \ac{infovis} deals with abstract data like for example movies in a movie database whereas \ac{scivis} operates on real-world data with spatial character.

These kind of visualizations are also already used in practical applications. The following list shows some summarized examples of these applications:
\begin{description}

\item[Urban Planning] \newline
Urbanists use \ac{geovis} to "[\ldots] model environmental interests and policy concerns of the general public" \iacite{Jiang2003}. \citeauthor{Jiang2003} also mention two examples, in which "[\ldots] 3D photorealistic representations are used to show urban redevelopment and dynamic computer simulations are used to show possible pollution diffusion over the next few years".

\citeauthor{Jiang2003} also describe that the widespread use of the internet by the general public allows collaborative planning to be conducted in both centralised and decentralised manner. The former way of planning in committee rooms or computer facility rooms can be extended with a decentralised web platform. This platform would lead to increased parcitipation by the public because
\begin{enumerate}
\item the internet already integrates various interactive and proactive techniques and
\item is time and place independent.
\end{enumerate}


\end{description}


Traditional, static maps have a limited exploratory capability. The graphical representations are inextricably linked to the geographical information beneath. \ac{gis} and \ac{geovis} allow for more interactive maps. Both of them have the ability to extend the basic layer of a map with user-experience abilities like for example zooming in or out and to change the visual appearance of the map \iacite{Jiang2003}.