\isection{Abstract}
In this paper the plan to create an evaluation framework for multi-agent cooperative games is presented.\\*
Dynamic difficulty adjustment is the technique of adapting the games difficulty to make it optimally challenging -- neither too boring, nor too stressful -- for the player. Performing an adjustment online means to execute it while the game is running rather than between games. Making adjustments that yield to optimal challenge is usually unequally hard for different types of games.\\*
To evaluate appropriate algorithms the General Video Game Artificial Intelligence framework is an ideal testbed. For many years artificial intelligence was exclusively designed to solve particular problems in known environments and by using specific heuristics and domain knowledge to achieve the goal.\\*
Recently there is an increase to make artificial intelligence agents independent from the domain used in. Those should be able to perform well in different contexts even ones not designed for. To make this possible competitions use so called game description languages. Those describe the environment and the dynamic, i.e. the interaction of components in the game. The General Video Game Artificial Intelligence framework uses the standardised Video Game Description Language  and provides an interface to create a general video controller with access to general information of the game like current time, score, termination conditions (win, loose), position of all components in the world, a list of game events (collisions, usage of items) and the list of available actions. The same information in different games described with such a language may require the controller to take different actions.\\*
The framework will be adapted to allow creating controller to be single-cooperative, i.e. it does not adapt to other artificial intelligence agents but to the player.

\paragraph{Keywords:}
\textit{Artificial General Intelligence, Dynamic Difficulty Adjustment, GVG-AI}