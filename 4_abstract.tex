\section*{Abstract}
\vspace{0.5cm}

There has never been such a large amount of data available in history and the global amount of information has never grown so fast and continuous. Processing and analysis of data garners increased economic and scientific interest. Finding useful and usable information in a multitude of data is a tedious and time consuming task. Visualisations are a effective tool to explore and present unknown data. However, there are many different types of visualisations possible to achieve one objective. Choosing the best visual representation is often based on the domain knowledge of the creator. They are often unaware of the strengths and weaknesses of each type of visualisation and therefore also its use cases.
\cbstart
Animations are one way of showing the creation of a visualisation in an effective way. Being able to follow the emergence of it could affect the comprehensibility and readability. Thus, this thesis presents a tool where animated transitions between different visual representations try to reveal their advantages and disadvantages. In order to evaluate the tool, a user study is conducted which should yield results of the effectiveness of such transitions.
\cbend
\vspace{0.5cm}
\textbf{\textit{Keywords:}} \\
\textit{Visualisation, Animation, Transition, Animated Transition}