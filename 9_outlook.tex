\section{Conclusion and Future Work}
\label{s:outlook}

% TODO: intro mit "animationskonzept erwaehnen und ziel der arbeit erwaehnen"

The user study conducted for this thesis showed potential for animated transitions. The default speed and order of transitions are not yet ideal. Follow-up studies could use different orders of transitions to get more information and results.

Furthermore, as already mentioned in Chapter \ref{s:discussion-study} on page \pageref{s:discussion-study}, performing a study with a different study design and more specific tasks would yield more significant results compared to the study conducted in this thesis. A different study design would have yielded results with stronger significance. Using a between subject study design could yield results with better comparability among each other. Furthermore, participants should be set a task for each transition type. This would result in more detailed findings, for instance the effectiveness of each transition. In addition, other studies could adapt the order of the transitions used from one visualisation type to another.

Moreover, the feedback of the conducted user study also suggested some implementation features. Some participants wanted to have more control over the animation itself. They demanded at least a play and pause function to keep track of particular spots during the animated transition. Implementing such features could significantly affect the results of the study, especially the transition score could yield better outcomes. Besides some implementation suggestions concering animation control, some participants also wanted to draw a comparison between the implemented in-place transition and non-in-place transitions. Due to the modularity provided by the transition manager and the component-based architecture of the whole application, such a request could be realised with ease without changing the current architecture.

Chapter \ref{s:discussion-study} on page \pageref{s:discussion-study} also mentions that answering the generalised question of this thesis is not possible with a single use case tested. In order to say which tasks can be supported by particle aggregation in a thematic map more and different use cases are needed. In addition, testing a variety of various animated transitions for changing the visual appearance would also yield in more information and data which could furthermore be used to derive answers according to the question.

In summary, the practical implementation still needs some improvements and refinement related to timing of the animations and based on further feedback, studies using different datasets could already yield diverging and useful results overall.
