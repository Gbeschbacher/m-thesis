\section{Conclusion and Future Work}
\label{s:outlook}
\cbstart
The user study conducted for this thesis showed potential for animated transitions. In order to get more significant results the study would need another study design with more specific tasks (as already mentioned in Chapter \ref{s:discussion-study} on page \pageref{s:discussion-study}). Feedback of the conducted user study also suggested some implementation features. Some participants wanted to have more control over the animation itself. They demanded at least a play and pause function to keep track of particular spots during the animated transition. Implementing such features could significantly affect the results of the study, especially the transition score could yield better outcomes. Besides some implementation suggestions concering animation control, some participants also wanted to draw a comparison between the implemented in-place transition and non-in-place transitions. Due to the modularity provided by the transition manager and the component-based architecture of the whole application, such a request could be realised with ease without changing the current architecture.

Chapter \ref{s:discussion-study} on page \pageref{s:discussion-study} also mentions that answering the generalised question of this thesis is not possible with a single use case tested. In order to say which tasks can be supported by particle aggregation in a thematic map more and different use cases are needed. In addition, testing a variety of various animated transitions for changing the visual appearance would also yield in more information and data which could furthermore be used to derive answers according to the question.

All in all, the practical implementation still needs some improvements related to timing of the animations and based on further feedback, studies using different datasets could already yield diverging and useful results overall.
\cbend
