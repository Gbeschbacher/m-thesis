\section{Outlook}
\label{s:outlook}
Chapter \ref{s:discussion-study} on page \pageref{s:discussion-study} already mentions that answering the generalised question of this thesis is not possible with a single use case tested. In order to say which tasks can be supported by particle aggregation in a thematic map more and different use cases are needed. In addition, testing a variety of various animated transitions for changing the visual appearance would also yield in more information and data which could furthermore be used to derive answers according to the question.

Feedback of the conducted user study also suggested some implementation features. Some participants wanted to have more control over the animation itself. They demanded at least a play and pause function to keep track of particular spots during the animated transition. Implementing such features could significantly affect the results of the study, especially the transition score could yield better outcomes. Besides some implementation suggestions concering animation control, some participants also wanted to draw a comparison between the implemented in-place transition and non-in-place transitions. Due to the modularity provided by the transition manager and the component-based architecture of the whole application, such a request could be realised with ease without changing the current architecture.

Another proposal for future work is testing the current setup with different datasets which could yield diverging results.

% as the user feedback already showed in the study, implementing more interaction possibilites for the animation could yield to better results overall
% furthermore non in-place transitions could be tried
% also the same setup could be tested with different datasets

% wie bereits erwaeht, koennten noch eine menge anderer transitionen implementiert werden wie z.B. jede art von non-inplace transitions, die vl. bessere ergebnisse bringen koennte als die implementierten in-place transitions; hierbei sind der kreativitaet keine grenzen gesetzt


% andere datensets / transitions ausprobieren fuer vergleichbare ergebnisse der konkreten fragestellung und
% andere use-cases oder loesungsansaetze probieren fuer die generalisierte fragestellung



% for outlook maybe mention something in the line of "the user study showed potential for transitions, but still needs improvements. based on the feedback, further studies could do ...". would make it maybe a bit more relatable