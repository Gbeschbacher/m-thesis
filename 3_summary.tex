\selectlanguage{ngerman}

\section*{Kurzfassung}
\vspace{0.5cm}

Die weltweit verfügbare Menge an Daten steigt kontinuierlich an. Heute stehen Daten in einem Umfang zur Verfügung wie nie zuvor. Die Verarbeitung und Analyse großer Datenmengen steigert auch das ökonomische und wissenschaftliche Interesse. Allerdings ist das Finden von nutzbaren und sinnvollen Daten in solchen Mengen ein sehr mühsames und zeitintensives Unterfangen. Visualisierungen sind ein praktisches Werkzeug für explorative Analyse unbekannter Daten. Jedoch können viele verschiedene Arten von Visualisierungen erstellt werden, um ein konkretes Ziel zu erreichen. Die Auswahl der Visualisierung für eine Aufgabe ist meist abhängig von dem Domänenwissen der Ersteller. Dadurch sind die Stärken und Schwächen der verschiedenen Arten von visuellen Repräsentationen oftmals nicht bekannt. Hauptaufgabe dieser Arbeit ist es herauszufinden, ob es möglich ist mit Hilfe von animierten Übergängen zwischen verschiedenen Visualisierungen die Vorteile beziehungsweise Nachteile einzelner Darstellungen aufzudecken. Die dafür implementierte Anwendung ist öffentlich verfügbar und die Nützlichkeit des Systems wird mit Hilfe einer Nutzerstudie getestet.

\vspace{0.5cm}
\textbf{\textit{Schlagwörter:}}\\
\textit{Visualisierungen, Animationen, Übergänge, animierte Übergänge}

\selectlanguage{english}
