\usepackage{textcomp}
\usepackage{xcolor}
\usepackage{listings}

% --- color definitions ---
\definecolor{mktext}    {RGB}{000, 043, 054}
\definecolor{mktarget}  {RGB}{211, 054, 130}
\definecolor{mkstring}  {RGB}{000, 127, 127}
\definecolor{mkvariable}{RGB}{038, 139, 210}
\definecolor{mkkeyword} {RGB}{133, 153, 000}
\definecolor{mklinecont}{RGB}{220, 050, 047}
\definecolor{mkcomment} {RGB}{088, 110, 117}
\definecolor{mkcommand} {RGB}{042, 161, 152}

% --- auxiliary macros for styles ---
\newcommand\mktextstyle{\color{mktext}}
\newcommand\mktargetstyle{\color{mktarget}}
\newcommand\mkstringstyle{\color{mkstring}}
\newcommand\mkvariablestyle{\color{mkvariable}}
\newcommand\mkkeywordstyle{\color{mkkeyword}}
\newcommand\mklinecontstyle{\color{mklinecont}}
\newcommand\mkcommentstyle{\slshape\color{mkcomment}}
\newcommand\mkcommandstyle{\color{mkcommand}}

\makeatletter
% --- context switches ---
\newif\iftarget@make@
\newif\iflinecont@make@

% --- hook into listings to keep track of context ---
\lst@AddToHook{InitVarsBOL}
{%
  \iflinecont@make@%
    \global\target@make@false%
  \else
    \global\target@make@true%
  \fi
  \global\linecont@make@false%
}

\lst@AddToHook{Output}
{%
  \ifnum\lst@mode=\lst@Pmode%
      \iftarget@make@%
        \gdef\globalstyle@make{\the@lststyle}%
        \def\lst@thestyle{\mktargetstyle}%
      \fi
    \fi
  \ifnum\lst@mode=\lst@Pmode% only test for keyword if we're in "processing" mode
    \lsthk@DetectKeywords%
  \fi
}

\lstdefinestyle{make-pretty}
{
  upquote=true,
  showstringspaces=false,
  basicstyle=\ttfamily\scriptsize\mktextstyle,
  sensitive=true,
  alsoletter=.-,
  keywords = {.CC,.CFLAGS,.PHONY,.SUFFIXES,SUFFIX,CPP_},
  keywordstyle = \mkkeywordstyle,
  moredelim=[is][\mkvariablestyle(\aftergroup\closingparen@make]{(}{)},
  moredelim=*[l][\mkcommandstyle]{\^^I},
  morestring=[b]',
  morestring=[b]",
  stringstyle=\mkstringstyle,
  literate =
    {\%}{{{\mkvariablestyle\%}}}1
    {\$}{{{\mkvariablestyle\$}}}1
    {\$*}{{{\mkvariablestyle\$*}}}2
    {\\}{{{\mklinecontstyle\lstum@backslash}}}1,
  MoreSelectCharTable =
    \processColon@make
    \processEqual@make
    \processBackslash@make
    \processTab@make,
  morecomment = [l]{\#},
  commentstyle = \mkcommentstyle,
}

% --- helper macros ---
\newcommand\closingparen@make{{\mkvariablestyle)}}

\newcommand\processEqual@make
{%
  \lst@DefSaveDef{`=}\equal@make%
  {%
    \equal@make%
    \global\target@make@false%
    \def\lst@thestyle{\globalstyle@make}%
  }%
}

\newcommand\processColon@make
{%
  \lst@DefSaveDef{`:}\colon@make%
  {%
    \colon@make%
    \global\target@make@false%
    \def\lst@thestyle{\globalstyle@make}%
  }%
}

\newcommand\processBackslash@make
{%
  \lst@DefSaveDef{`\\}\backslash@make%
  {%
    \backslash@make%
    \ifnum\lst@mode=\lst@Pmode%
      \global\linecont@make@true%
      \def\lst@thestyle{\globalstyle@make}%
    \fi
  }%
}

\newcommand\processTab@make
{%
  \lst@DefSaveDef{`^^I}\tabchar@make%
  {%
    \tabchar@make%
    \global\target@make@false%
  }%
}
\makeatother