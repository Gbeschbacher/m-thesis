From a general point of view, a cartogram can be considered a special case of proportional symbol mapping. Instead of a symbol being scaled proportionally to a data magnitude, a cartogram scales its geographic areas. Thus a cartogram includes intentional distortion proportional to the value of an attribute. This also is the reason why some cartograms appear very similar to a map and some have little similarity to any kind of map. \citeauthor{Tyner2010} defines a cartogram as a geographic representation where size or distance is scaled to a variable other than earth size or distance units \iacite{Tyner2010}.
There are several types of cartograms. The most commonly seen are value-by-area cartograms. This kind of type is distinguished by the proportional size of enumeration units according to a value. Another type called distance or linear cartogram uses a time scale instead of a distance scale. Value-by-area cartograms are furthermore divided into three types \iacite{Tyner2010}:

\begin{enumerate}

\ditem{Contiguous Cartograms} \hfill \\
Borders between enumeration units are maintained as much as possible, although shapes are distorted. \citeauthor{Tyner2010} states that a contiguous cartogram approximating shape with straight line segments is the least confusing one to the reader. Shape is the essential factor for cartograms to preserve information. If individual states cannot be recognised and compared with a conventional map, then a cartogram will have no effect \iacite{Tyner2010}.

\ditem{Noncontiguous Cartograms} \hfill \\
Enumeration units are meaningful for this type of cartogram and they show shapes correctly. They either enlarge or reduce the units size according to the variable being mapped. Thus, enumeration units in a noncontiguous cartogram do not touch and are seperated by empty space \iacite{Tyner2010}.

\ditem{Variations} \hfill \\
A Dorling cartogram replaces the enumeration units with uniform abstract shapes, normally circles. It tries to place the shapes non-overlapping, whereas maintaining original shape or borders is not needed. In order to achieve this, the units are moved from their original location. One method of placing the shapes is by using some kind of collision detection and try to place them as close to their centroid as possible. The Demers cartogram is closely related to the Dorling cartogram. The only difference is the type of the shape. It uses a square to represent enumeration areas, which gives the advantage of permitting greater contiguity than circles \iacite{Tyner2010}.

\end{enumerate}

\newpage
A cartogram, in general, has two major weaknesses:
\begin{enumerate}
\item It distorts geography and therefore standard measurements, e.g. distance among places are not accurate anymore.
\item Without having the actual geographic shape of the map in mind, it is hard to interpret the cartogram correctly, because the sizes of shapes cannot be related to the enumeration areas anymore.
\end{enumerate}

However, despite their weaknesses, they also have a strong visual impact and therefore attract reader attention. This is definitely an advantage because it allows for a stronger impression of relative values compared to choropleth or dot density maps. On those type of maps, a high population in a small state is barely noticeable compared to a value-by-area cartogram.
Another minor advantage of a cartogram is its flexibility in data. There is no generalisation of data and therefore no loss of detail through it, as this would be the case in classification for choropleth maps \iacite{Tyner2010}.
