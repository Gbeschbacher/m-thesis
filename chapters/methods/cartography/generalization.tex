Generalization in a context with maps refers to "[\ldots] the selection, simplification and even symbolization of detail according to the purpose and scale of the map \iacite{Tyner2010}".
In general, it helps in communication and understanding of the objectives of the given map. If no generalization is applied, visual clutter would occur due to missing selectivity. \citeauthor{Tyner2010} says the main objective of this method is to emphasize the theme of a map, while still preserving important geographic patterns \iacite{Tyner2010}.

The following list shows an overview of the operations of generalization mentioned by \citeauthor{Tyner2010}. Every operation is described with a short summarization \iacite{Tyner2010}.

\begin{itemize}
\item \textbf{Selection} denotes the features on the map. This process can be subdivided into two tasks:

\begin{enumerate*}
\item Choosing categories of data like roads, railroads, etc. to be represented, and
\item choosing the amount of information within each category, e.g. showing only rivers with a certain size.
\end{enumerate*}

\item \textbf{Simplification} expresses the level of detail of the map. The details of coastlines, for example, is mostly unimportant for thematic maps. The basic graphical patterns need to be recognizable, but no other detail.

\item \textbf{Smoothing} can also be seen as part of the simplification process. A river with a lot of indentation and meanders can be smoothed with the result that the main characteristic (mostly direction) of the river is still shown.

\item \textbf{Grouping} means, that small features are often grouped together. In a simplistic example, single trees could be grouped into a forest, according to the level of detail and purpose of the map.

\item \textbf{Classification} for geographic data follows the same principle as categorizing data by a specific attribute.

\item \textbf{Exaggeration} includes small distortion of the map. If a narrow valley should show three different features, which would overlap, the valley will be widened.

\item \textbf{Displacement} of two very close, parallel roads makes them better distinguishable.

\item \textbf{Symbolization} is used for the selection and design of the symbol representing a phenomenon. It is not necessarily part of the generalization process and can also be seen as part of the selected thematic map.

\end{itemize}