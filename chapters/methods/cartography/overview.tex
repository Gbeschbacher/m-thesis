The origin of cartography lays far back in the history of visualisation as discussed in Chapter \ref{s:history} on page \pageref{s:history}. Today's understanding of modern cartography began in the late 18\textsuperscript{th} century with the attempt to show more than one attribute in a map. \citeauthor{Longley2005} say, that topography has long been understood as an important aspect of war strategy. Battles and wars could be decided on the information of the topology. A commander had to position his units wisely in order to exploit all geographical circumstances and thus have an advantage over his opponent \iacite{Longley2005}.
In the middle of the 20\textsuperscript{th} century, the Soviets, Fascist Italy and Nazi Germany all used maps to foster national pride. This could furthermore be used to justify their expansionism \iacite{Crampton2015}.
Today, national organisations produce a variety of maps for different reasons with a very high focus on map accuracy (see chapter \ref{s:map-accuracy} on page \pageref{s:map-accuracy} for more information) and transmittal of relevant information.

The field of cartography can be divided into two major subcategories:
\begin{enumerate}
\ditem{General cartography} is associated with maps that are constructed for commonalty. These type of maps often contain a variety of features and display many reference and location systems. An abstract definition of general maps is that those kinds of maps show the variety of phenomena of either geological, geographical or political nature together \iacite{Thrower2008}.

\ditem{Thematic cartography} focuses on a specific subject area, often called theme. Thus it involves maps which emphasize spatial variation of geographic distributions. \citeauthor{BartzPetchenik1979} describes thematic maps compared to general (or reference) maps as "in place, about space". She says that a general map is characterised by the fact that it shows you where something is in space, according to the example. However, thematic maps will tell a story about that specific place \iacite{BartzPetchenik1979}.
In order to make a connection to the abstract definition \citeauthor{Thrower2008} made for general maps, thematic maps could be described as follows:
Those kinds of maps (thematic) use the base data only as points of references. Base data could be coastlines, boundaries, and places. Phenomena of all kind are being mapped onto the reference \iacite{Thrower2008}.
\end{enumerate}

The following sections in this chapter provide general knowledge about different types of maps, scaling, projecting, generalization and symbolization and accuracy. This knowledge is needed in order to understand the implementation of the practical part of this thesis.

\subsubsection{Methods of Thematic Maps}
The two major categories of cartography are general cartography and thematic cartography, which are introduced in Chapter \ref{s:cartography} on page \pageref{s:cartography}. This categorisation can be directly adopted from cartography to maps. The main objective of this Chapter is to give an overview of different thematic maps and their usage. These maps can be subdivided into univariate and multivariate maps.

\cbstart
Univariate maps are only dependent on one variable, except for the map variables like latitude and longitude. The following Sections describe univariate thematic map visualisations in detail. Besides geo-spatial visualisations mapping a single attribute, there are also visualisations mapping multiple attributes. These are called multivariate types of visual representations. Each attribute on such a type of visualisation is mapped to a different visual channel. However, it is needed to consider the already used channel depending on the type of visualisation. As an example, a proportional symbol map already uses size As a visual channel and therefore a new attribute cannot use size as its visual channel.
Apart from the amount of attributes mapped to a geo-spatial visualisation, univariate and multivariate visual representations share their characteristics.
\cbend

\paragraph{Dot Density Map}
\label{s:dot}
The first dot map in history is shown in figure \ref{fig:cholera-map} on page \pageref{fig:cholera-map}. It was the first map of its kind and could help in the display of disease outbreaks. This type of univariate thematic map uses points or dots to map discrete data. Attribute values of the given data determine the number of dots displayed in a specific region. All dots need to be the same size. To explain the two different types of dot maps, imagine a dataset of customers where each customer has a location:

\begin{enumerate}
\ditem{One-to-one} \hfill \\
Each dot on the map represents exactly one item of the represented theme. Considering the example dataset mentioned above, every customer would be represented by exactly one dot.
\ditem{One-to-many} \hfill \\
Each dot on the map represents an aggregate of information. Therefore this type of dot map could be used with the aggregation of customers in a specific location. Thus the cartographer needs to decide how many customers are aggregated, or rather how many customers are represented by one dot.
\end{enumerate}

Both types of dot density maps share the purpose, that they are not a tool to determine exact quantities. Getting the exact amount of dots in a high density area is a very cumbersome task and users often tend to underestimate dot totals as density increases \iacite{McMaster2001}. However, it is a very common technique for viewing the clustering, dispersion, linearity, and general pattern of a distribution. The technique appeared first in the 19\textsuperscript{th} century and is today accepted as one of the primary techniques for representing geographic patterns \iacite{Tyner2010}.

The mapmaker can use dots in a dot density map with a different type of level of detail. This means, that dots do not necessarily need to have an exact location. If he or she wants to discover a pattern on a state-wide level of detail, dots can be placed anywhere in their corresponding states, as long as they do not leave their state boundaries.
Another location based decision which has to be considered is, if the dots should use some kind of pseudo-random placement in case of overlapping. This decision is based on a maximum overlap constraint. It can be thought of as a random placement of dots in a square without violating the constraint.

According to \citeauthor{Tyner2010}, there are some main design principles for dot maps that should be considered:
\begin{itemize}
\item The size of the dot.
\item The value assigned to the dot. This also includes the correct use of the two different types of dot maps.
\item The location of the dot on the map in case of an aggregated level of detail of the map.
\item The aggregated units in case of a one-to-many dot map. This design principle can be thought of as using a legend in order to tell the aggregated value one dot represents.
\end{itemize}
Changing any one of these can change the overall appearance and interpretation of the map \iacite{Tyner2010}.

The main advantage of this type of map is the readability. It requires little to no cognitive effort by the user to read the map when compared to other types. Specific advantages of dot maps are the good measure of density and the loose coupling between the size of a dot and its represented value.
However, reading specific information from those maps is not an easy task as mentioned before. Additionally, if a map uses some kind of random placement without any hint in the visualisation, map readers may potentially infer the locations of dots as precise locations of the mapped phenomenon. To counteract the second drawback, dot density maps with random placement of dots should consider the actual occurrence of the mapped phenomenon, e.g. dots should not be placed in lakes for a map of population.

\paragraph{Graduated Symbol or Proportional Symbol Map}
% TODO: do not start with an example

Figure \ref{fig:first-mixture} on page \pageref{fig:first-mixture} shows a special kind of a proportional symbol map. According to figure \ref{fig:va-channels} on page \pageref{fig:va-channels}, this type of map uses the visual channel of size to represent differences of discrete data. Again, this type of map can be subdivided into two categories: classed and unclassed. Classed ones are known as range-graded or graduated symbol maps, whereas unclassed ones are called proportional symbol maps. The latter one uses a symbol size proportional to the value of the attribute being mapped \iacite{Dutton.2014}.
Although circles are the most typical symbol used, it is possible to use any type of symbol, ranging from abstract, geometric symbols to pictographic symbols. Figure \ref{fig:different-symbols} on page \pageref{fig:different-symbols} shows two proportional symbol maps showing the same phenomenon. The left part of this figure uses the common circle as symbol, while the right side uses a pictogram. Albeit, \citeauthor{Dutton.2014} says, that squares or bars are easier to estimate the size of the symbol. However, the circle established because of its compactness due to its low perimeter to area ratio.

\begin{figure}[!htb]
\centering
\includegraphics[height=5cm,keepaspectratio]{images/psm/symbols.png}
\caption[
    Two types of proportional symbol maps with different symbols \iacite{Dutton.2014}.
]{Two types of proportional symbol maps with different symbols.}
\label{fig:different-symbols}
\end{figure}

Another consideration in terms of symbol used is the fact, that squares and bars tend to run off the page with large values earlier than circles might \iacite{Dutton.2014}. \citeauthor{FLANNERY1971} introduced a scaling factor for proportional circles for better estimation of the value. However, this correction may not be effective, because the correction itself does not consider the map context \iacite{FLANNERY1971}. A phenomenon related to the importance of context is known as the Ebbinghaus illusion. Figure \ref{fig:ebbinghaus} on page \pageref{fig:ebbinghaus} shows such an illusion. Both central circles actually have the same size, but because of the context of each side, the central circles appear different.

\begin{figure}[!htb]
\centering
\includegraphics[height=5cm,keepaspectratio]{images/psm/ebbinghaus.png}
\caption[
    Ebbinghaus illusion, Urldate: 07.2016 \newline
    \small\texttt{\url{https://upload.wikimedia.org/wikipedia/commons/b/bc/Mond-vergleich.svg}}
]{Ebbinghaus illusion}
\label{fig:ebbinghaus}
\end{figure}

In order to combat the problem of value estimation, there are two major choices:
\begin{enumerate}
\item A legend could show proportional symbols which represent the different values of the mapped phenomenon. One possibility would be to display the smallest symbol, the largest symbol and some symbols at intermediate values.
\item Another alternative is to use range-graded symbols. Therefore the data needs to be classified, but in exchange the estimation problem is completely avoided. This alternative still needs consideration in the size of symbols, because each symbol should still be distinguishable from each other.
\end{enumerate}

Based on the given knowledge about proportional and gradient symbol maps, it is possible to derive some main design principles:
\begin{itemize}
\item The estimation of the value of a symbol is key for this type of map. This is most easily accomplished with geometric symbols.
\item Use a legend with examples to increase the reader's ability to correctly estimate the value of a symbol.
\end{itemize}

A close related problem to the user's estimation problem is the actual scaling technique used. According to \citeauthor{Dent2008} the three most commong techniques used are
\begin{enumerate*}[label={(\arabic*)]
\item absolute scaling,
\item apparent magnitude scaling and
\item range grading \iacite{Dent2008}.
\end{enumerate*}

\begin{enumerate}
\ditem{Absolute scaling} makes each symbol fit its data value on the scale being used. This means, that a symbol representing four items in a dataset is twice as big as a symbol representing two.
\ditem{Apparent magnitude scaling} compensates for human error interpretation in scale. Using this technique, a symbol having twice as much in value is not twice as big, because it would appear smaller, leading to interpretation error. This type of scaling takes this error into account and increased the size of a symbol by more than the proportional amount \iacite{Krygier.2007}.
\ditem{Range grading} classifies the data into a fixed amount of groups. Each group has a fixed range of values and the same symbol to represent. The groups only differ in the range of values they represent and the size of the symbols.
\end{enumerate}

The main advantage of a proportional symbol map is the flexibility it provides. Possible data can either be of numerical or categorical nature. Even the way the data is used is adjustable. An item can be mapped on a precise location or to geographic areas, depending on the level of detail.
Comparing proportional symbol maps with dot density maps, one advantage is observable: the estimation problem of dot density maps becomes less tedious when using proportional symbols. However, if proportional symbol maps are put in comparison with choropleth maps, they also have an advantage: the size of the enumeration unit does not matter. This problem will be explained in detail in chapter \ref{s:choropleth} on page \pageref{s:choropleth}.
A major drawback of this type of map is symbol overlapping, especially if locations are close together. Using transparency allows at least overlapped symbols to show through. Another possibility to combat this issue is to move symbols apart manually. However, this runs the risk of disassociating the location with its symbol.

\paragraph{Choropleth Map}
For continuous data, two mapping techniques are commonly used: choropleth and isarithmic mapping. This chapter will only cover choropleth mapping because the practical part of this thesis will not feature isarithmic mapping. The basic idea of choropleth mapping is applying value or color intensity based on some statistical value to enumeration units like census tracts, counties, states or nations. The higher a value assigned to an enumeration unit, the more saturated the color of that unit. Fundamental to this type of mapping is standardization and classification of raw data. If this is not the case and raw data is used, the visualization will suffer from an inherent areal bias \iacite{McMaster2010}. This problem is best described with a practical example: the united kingdom has greater population but lower area compared to canada. However, if the mapped data is not standardized by area, canada will have a higher visual impact than united kingdom due to their superior areal extent, even though the attribute mapped onto canada has a lower value compared to the one on united kingdom. Therefore the less the mapped attribute is tied to enumeration units, the less sense a choropleth map makes.

The following list shows the most rudimentary and most common methods to classify data mentioned by \citeauthor{McMaster2010}. The amount of classes must be known before and to explain the classifications below, five different classes with a value range from 10 to 85 is used.

\begin{itemize}

\item \textbf{Equal interval classification} assumes equal range between the class breaks. With the mentioned example, class one would include values from 10 to 25, class two would include values from 25 to 40, and so forth \iacite{McMaster2010}.

\item \textbf{Quantiles classification} needs to know the amount of items in a dataset in addition to the amount of desired classes. Consider 100 observations in a dataset and five desired classes. Thus, 20 observations will be placed in each class. These 20 observations can be thought of as data values. Putting this amount of data values in five different classes results in 4 data values per class. Now it is only needed to go through the whole dataset and put the first to the fourth item in the first class, the fifth to the ninth item in the second class, and so forth \iacite{McMaster2010}.

\item \textbf{Natural breaks classification} follows the idea of minimizing the internal variation of the dataset, while maximizing the variation among the classes. The user needs to choose significant gaps in the dataset, according to the number of desired classes \iacite{McMaster2010}.

\end{itemize}

In order to create meaninful choropleth maps, some design principles need to be considered:

\begin{itemize}

\item The visual channel this type of mapping is based on should always be a sequential or diverging color scheme for continuous data, depending on wheter there is a meaningful zero point or not. Figure \ref{fig:colorbrewer} on page \pageref{fig:colorbrewer} shows three different categories of color scales. The reason why choropleth maps should only use diverging or sequential scales is, because these two types are perceived as ordered. Due to their brightness and saturation, they can be ordered from low to high or vice versa, whereas qualitative scales are perceptually nonlinear. Those scales are best used for categorical data.

\item Choropleth maps are commonly used with data representing derived quantities. Examples of derived quantities are density, average, rate and percentage. If this type of map is not used with such data, it has to be read with caution. Aggregated data e.g. crimes committed or amount of orders can also be used, if the areal bias is taken into consideration when interpreting the map.

\item A choropleth map without a legend is not negotiable, according to \citeauthor{Dent2008}. All colors on the map represent a specific value or class, which is defined and explained in the legend. All classifications would be pointless without a map legend \iacite{Dent2008}.

\end{itemize}

\begin{figure}[!htb]
\centering
\includegraphics[height=5cm,keepaspectratio]{images/choropleth/color-scales.png}
\caption[
    Three different categories of color scales, Urldate: 07.2016 \newline
    \small\texttt{\url{http://www.gnuplotting.org/figs/colorbrewer.png}}
]{Three different categories of color scales.}
\label{fig:colorbrewer}
\end{figure}

Until now, only so called classed choropleth maps were decribed. Unclassed choropleth maps follow the idea of "letting the data speak for itself". These specific type of a choropleth map assigns a unique color to each unique data value without previous classification. They also make use of a continuous color scale. The major difference to classed ones is best described with an example: consider a dataset consisting of unemployment rates for each state in the united states. If there is a big numerical gap between the second and third highest unemployment rate for two states, their corresponding color would also have a significant jump. Thus the data is placed proportionally along the color scale.

The main objective of choropleth maps is to depict the geographic distribution of the data magnitudes. Ideally the choice of fill will communicate the range from low data magnitudes to high magnitudes through an obvious change from light to dark. If an overall geographic pattern should be shown in the map, an unclassed choropleth map is the way to go. If that is not the case and locations should be comparable against each other, classed choropleth maps should be used.



\label{s:choropleth}

\paragraph{Cartogram}
From a general point of view, a cartogram can be considered a special case of proportional symbol mapping. Instead of a symbol being scaled proportionally to a data magnitude, a cartogram scales its geographic areas. Thus a cartogram includes intentional distortion proportional to the value of an attribute. This also is the reason why some cartograms appear very similar to a map and some have little similarity to any kind of map. \citeauthor{Tyner2010} defines a cartogram as a geographic representation where size or distance is scaled to a variable other than earth size or distance units \iacite{Tyner2010}.
There are several types of cartograms. The most commonly seen are value-by-area cartograms. This kind of type is distinguished by the proportional size of enumeration units according to a value. Another type called distance or linear cartogram uses a time scale instead of a distance scale. Value-by-area cartograms are furthermore divided into three types \iacite{Tyner2010}:

\begin{enumerate}

\ditem{Contiguous Cartograms} \hfill \\
Borders between enumeration units are maintained as much as possible, although shapes are distorted. \citeauthor{Tyner2010} states that a contiguous cartogram approximating shape with straight line segments is the least confusing one to the reader. Shape is the essential factor for cartograms to preserve information. If individual states cannot be recognized and compared with a conventional map, then a cartogram will have no effect \iacite{Tyner2010}.

\ditem{Noncontiguous Cartograms} \hfill \\
Enumeration units are meaningful for this type of cartogram and they show shapes correctly. They either enlarge or reduce the units size according to the variable being mapped. Thus, enumeration units in a noncontiguous cartogram do not touch and are seperated by empty space \iacite{Tyner2010}.

\ditem{Variations} \hfill \\
A Dorling cartogram replaces the enumeration units with uniform abstract shapes, normally circles. It tries to place the shapes non-overlapping, whereas maintaining original shape or borders is not needed. In order to achieve this, the units are moved from their original location. One method of placing the shapes is by using some kind of collision detection and try to place them as close to their centroid as possible. The Demers cartogram is closely related to the Dorling cartogram. The only difference is the type of the shape. It uses a square to represent enumeration areas, which gives the advantage of permitting greater contiguity than circles \iacite{Tyner2010}.

\end{enumerate}

A cartogram, in general, has two major weaknesses:
\begin{enumerate}
\item It distorts geography and therefore standard measurements, e.g. distance among places are not accurate anymore.
\item Without having the actual geographic shape of the map in mind, it is hard to interpret the cartogram correctly, because the sizes of shapes cannot be related to the enumeration areas anymore.
\end{enumerate}

However, despite their weaknesses, they also have a strong visual impact and therefore attract reader attention. This is definitely an advantage because it allows for a stronger impression of relative values compared to choropleth or dot density maps. On those type of maps, a high population in a small state is barely noticeable compared to a value-by-area cartogram.
Another minor advantage of a cartogram is its flexibility in data. There is no generalization of data and therefore no loss of detail through it, as this would be the case in classification for choropleth maps \iacite{Tyner2010}.



\subsubsection{Map Scale}
\label{s:map-scale}
Maps are scale models of the earth by itself. Scaling is a simple concept per se but is complicated by the earth's curvature. It controls multiple things e.g. how much data can be mapped in a map frame, the size of symbols, the overlap of symbols, and so forth \iacite{Longley2005}. Without considering curvature, scaling can be explained as the reduction or enlargement of real world objects by a constant amount.

The following list shows two different representations of scales:
\begin{enumerate}
\ditem{Bar scale} \hfill \\
Figure \ref{fig:bar-scale} on page \pageref{fig:bar-scale} visually shows the scale of a map. The first black bar indicates, that $1$cm in the map represents $100$cm in real world. It includes everything needed: the ratio, the unit of measure and a visual example of both.

\begin{figure}[!htb]
\centering
\includegraphics[width=0.4\textwidth,keepaspectratio]{images/methods/scalebar.jpg}
\caption[
    Bar scale.
]{Bar scale.}
\label{fig:bar-scale}
\end{figure}

\ditem{Lexical scale} \hfill \\
Words describing a ratio are known as lexical scales. $1:1000$ can be interpreted as as scale, where $1$ unit of measure on the map, represents $1000$ units of measure in real world. The unit of measure is very important to mention in a lexical scale.
\end{enumerate}

However, taking the earth's curvature into consideration, scaling large areas result in noticeable distortions. The distribution of distortion is dependent on the map projection (see Chapter \ref{s:map-projections} on page \pageref{s:map-projections} for more information).

Choosing the correct type of scale depends on several factors: the theme and purpose of the map, data resolution, and the specified format. If a map is designed for navigation, it needs more detail than if it is designed to show an overview of a national park \iacite{Tyner2010}.

\subsubsection{Map Projections}
As chapter \ref{s:map-scale} on page \ref{s:map-scale} already mentioned, the map scale is heavily dependent on the map projection. The true figure of the earth is not a regular shape like a sphere or an ellipsoid. It has a seperate shape, which is called geoid.
From a general point of view, every projection distorts the earth somehow, but the advantage of projection is, that any point can be exactly recreated at any given time, due to the fact, that the projection affects the whole earth.
A projection consists of four main properties:
\begin{enumerate*}
\item area,
\item form,
\item distance and
\item directions.
\end{enumerate*}
Every projection affects all its properties in some way. Some of them preserve area and form while distorting distances and directions \iacite{Snyder1987}.

In order to understand the sub-chapters explaining different types of projections, some terms need to be explained first:

\begin{enumerate}

\ditem{Conformal} \hfill \\
If a projection is conformal, it preserves local angles in the map. This can be thought of preserving the general shape of e.g. an island. Some parts of an island may get larger or smaller due to a conformal projection, but the recognizability of the island is still given \iacite{Snyder1987}.

\ditem{Loxodromes} \hfill \\
According to the Merriam Webster Online Dictionary\footnote{See \href{http://www.merriam-webster.com/}{Merriam Webster Dictionary}} a loxodrome is also called rhumb line and can be defined as "[\ldots] a line on the surface of the earth that follows a single compass bearing and makes equal oblique angles with all meridians". A more practical example of loxodromes is to imagine a sailing route between two points. This line is shown as a straight line, as long as the intended course of the ship remains constant with respect to north.

\ditem{Equal-Area} \hfill \\
If a map bares the name equal-area, it means that it preserves area by distorting shapes.

\end{enumerate}

For most thematic maps irregularities of the earth's shape are ignored, as long as geodetic accuracy is not related to the purpose of the map \iacite{Snyder1987}. Figure \ref{fig:projections-base} on page \ref{fig:projections-base} is in the subsections and helps to explain some major characteristics of the different types of projections. Every projection is only listed with its characteristics. These will not be discussed in detail, as it would go beyond the scope of this thesis.

\begin{figure}[!htb]
\centering
\includegraphics[width=0.8\textwidth,keepaspectratio]{images/methods/mappings.png}
\caption[
    Projection of the earth onto the three major surfaces \iacite{Snyder1987}.
]{Projection of the earth onto the three major surfaces.}
\label{fig:projections-base}
\end{figure}

\paragraph{Cylindrical map projections}
The main concept of cylindrical map projections consist "[\ldots] of meridians which are equidistant parallel straight lines, crossed at right angles by straight parallel lines of latitude, generally not equidistant \iacite{Snyder1987}".
In general, cylindrical map projections can be thought of unrolling a cylinder which has been wrapped around a geoid, touching at the equator (see figure \ref{fig:projections-base} on page \pageref{fig:projections-base}).
The primary use for this type of projection is to either map the complete world, or for maps along narrow strips of a great circlel arc, such as the equator.

The following list will only feature two common projections accordingly to cylindrical projections with a short description taken from \citeauthor{Snyder1987} \iacite{Snyder1987}:

\begin{enumerate}

\ditem{Mercator Projection} \hfill \\
It is a cylindrical, conformal projection where meridians are equally spaced straight lines, whereas parallels are unequally spaced straight lines. The scale of the map is only true along the equator and loxodromes are straight lines. The biggest distortion appears close to the poles. The major advantage according to \citeauthor{Snyder1987} is the navigational feature that loxodromes are straight lines \iacite{Snyder1987}. Figure \ref{fig:projections-mercator} on page \pageref{fig:projections-mercator} shows an example mercator projection of the earth.

\begin{figure}[!htb]
\centering
\includegraphics[height=5cm,keepaspectratio]{images/methods/projections/mercator.png}
\caption[
    Mercator projection, Urldate: 07.2016 \newline
    \small\texttt{\url{https://upload.wikimedia.org/wikipedia/commons/f/f0/MercNormSph.png}}.
]{Mercator projection}
\label{fig:projections-mercator}
\end{figure}


\ditem{Transverse Mercator Projection} \hfill \\
This projection is similar to the basic mercator projection except the main difference of a transverse, cylindrical and conformal projection. The central meridian and each meridian 90 degrees east and west of the central meridian are straight lines. All other meridians and parallels are complex curves. The scale of the map is only true along the central meridian \iacite{Snyder1987}. Figure \ref{fig:projections-mercator-transverse} on page \pageref{fig:projections-mercator-transverse} illustrates the described characteristics.

\begin{figure}[!htb]
\centering
\includegraphics[height=5cm,keepaspectratio]{images/methods/projections/mercator-transverse.png}
\caption[
    Transverse mercator projection, Urldate: 07.2016 \newline
    \small\texttt{\url{https://upload.wikimedia.org/wikipedia/commons/1/15/MercTranSph.png}}.
]{Transverse mercator projection}
\label{fig:projections-mercator-transverse}
\end{figure}

\end{enumerate}

\paragraph{Conic map projections}
Conic projections are preferred over cylindrical ones if the purpose of the map is to show a region for which the greatest areal extent is from east to west in the temperature zone. This projection type makes use of "[\ldots] arcs of concentric circles for parallels of latitude and equally spaced straight radii of these circles for meridians \iacite{Snyder1987}." The main distinctive feature is based on placing a cone on the top of a globe representing the earth (see figure \ref{fig:projections-base} on page \pageref{fig:projections-base}).

The following list will only feature two common projections with a short description taken from \citeauthor{Snyder1987} \iacite{Snyder1987}:

\begin{enumerate}
\ditem{Albers Equal-Area Projection} \hfill \\
This projection, as seen in figure \ref{fig:projections-albers-ea} on page \pageref{fig:projections-albers-ea}, is a conic, area-preserving one where parallels are unequally spaced arcs of concentric circles, whereas meridians are equally spaced radii of the same circles. It features no distortion in scale or shape along two standard parallels, normally, or along just one. Both poles are arcs of circles. The projection is used for regions with predominant east-west expanse \iacite{Snyder1987}.

\begin{figure}[!htb]
\centering
\includegraphics[height=5cm,keepaspectratio]{images/methods/projections/albers.jpg}
\caption[
    Albers equal-area projection, Urldate: 07.2016 \newline
    \small\texttt{\url{https://upload.wikimedia.org/wikipedia/commons/1/1f/Albers_projection_SW.jpg}}.
]{Albers equal-area projection}
\label{fig:projections-albers-ea}
\end{figure}

\ditem{Equidistant Conic Projection} \hfill \\
Equidistant conic projection displays parallels, including poles, as arcs of concentric circles evenly spaced along the meridians. Like the albers equal-area projection, equidistant projection also displays meridians as equally spaced radii of the same circles and thereby cutting parallels at right angles. The scaling of the map is true along all meridians and along one or two standard parallels \iacite{Snyder1987}. Figure \ref{fig:projections-equidistant} on page \pageref{fig:projections-equidistant} shows similarities to the albers equal-area projection. The major distinction is the distortion of direction, area and shape according to the distance from standard parallels.

\begin{figure}[!htb]
\centering
\includegraphics[height=5cm,keepaspectratio]{images/methods/projections/equidistant.jpg}
\caption[
    Equidistant conic projection, Urldate: 07.2016 \newline
    \small\texttt{\url{https://upload.wikimedia.org/wikipedia/commons/d/d8/Equidistant_conic_projection_SW.JPG}}.
]{Equidistant conic projection}
\label{fig:projections-equidistant}
\end{figure}

\end{enumerate}
\paragraph{Azimuthal and related map projections}
Even though cylindrical and conic projections are related to cylinders and cones wrapped around the globe, azimuthal projections are mapped onto a plane. This plane usually is placed tangential at either pole, the equator, or any intermediate point. Each placement bears a different name and are called polar, equatorial and oblique aspects respectively. This type of projection attracted attention with the rise of radio transmission. This is due to the fact, that those type of projections show the direction from the center of the projection to any other point on the map correctly. Figure \ref{fig:projections-azimuthal} on page \pageref{fig:projections-azimuthal} illustrates a polar azimuthal projection. All meridians are straight lines and radiate at their true angles from the center, whereas parallels are concentric circles. Most azimuthal projections do not have standard parallels or standard meridians, because each map has only one standard point. Azimuthals are not suitable for regions with predominant expanse in one direction, because they will maximize distortion \iacite{Snyder1987}.

\begin{figure}[!htb]
\centering
\includegraphics[height=5cm,keepaspectratio]{images/methods/projections/equidistant.jpg}
\caption[
    Azimuthal projection, Urldate: 07.2016 \newline
    \small\texttt{\url{https://upload.wikimedia.org/wikipedia/commons/e/ec/Azimuthal_equidistant_projection_SW.jpg}}.
]{Azimuthal projection}
\label{fig:projections-azimuthal}
\end{figure}
\label{s:map-projections}

\subsubsection{Map Generalization}
Generalization in a context with maps refers to "[\ldots] the selection, simplification and even symbolization of detail according to the purpose and scale of the map \iacite{Tyner2010}".
In general, it helps in communication and understanding of the objectives of the given map. If no generalization is applied, visual clutter would occur due to missing selectivity. \citeauthor{Tyner2010} says the main objective of this method is to emphasize the theme of a map, while still preserving important geographic patterns \iacite{Tyner2010}.

The following list shows an overview of the operations of generalization mentioned by \citeauthor{Tyner2010}. Every operation is described with a short summarization \iacite{Tyner2010}.

\begin{itemize}
\item \textbf{Selection} denotes the features on the map. This process can be subdivided into two tasks:

\begin{enumerate*}[label={(\arabic*)}]
\item Choosing categories of data like roads, railroads, etc. to be represented, and
\item choosing the amount of information within each category, e.g. showing only rivers with a certain size.
\end{enumerate*}

\item \textbf{Simplification} expresses the level of detail of the map. The details of coastlines, for example, is mostly unimportant for thematic maps. The basic graphical patterns need to be recognizable, but no other detail.

\item \textbf{Smoothing} can also be seen as part of the simplification process. A river with a lot of indentation and meanders can be smoothed with the result that the main characteristic (mostly direction) of the river is still shown.

\item \textbf{Grouping} means, that small features are often grouped together. In a simplistic example, single trees could be grouped into a forest, according to the level of detail and purpose of the map.

\item \textbf{Classification} for geographic data follows the same principle as categorizing data by a specific attribute.

\item \textbf{Exaggeration} includes small distortion of the map. If a narrow valley should show three different features, which would overlap, the valley will be widened.

\item \textbf{Displacement} of two very close, parallel roads makes them better distinguishable.

\item \textbf{Symbolization} is used for the selection and design of the symbol representing a phenomenon. It is not necessarily part of the generalization process and can also be seen as part of the selected thematic map.

\end{itemize}

\subsubsection{Map Accuracy}
\label{s:map-accuracy}
A map's accuracy should mainly be determined by its purpose. It is usually taken to mean positional or locational accuracy. A map, which is strictly positional accurate, means that all features displayed have the correct scale and are precise in their location. However, thematic maps are often drawn at a very small scale, making it impossible to achieve such an accuracy. Generalization impacts accuracy heavily. If generalization with a focus on certain aspects of a map is applied, the map's accuracy will suffer \iacite{Tyner2010}.

According to \citeauthor{Tyner2010}, it is still possible to be rather truthful and accurate. This can be achieved by establishing following mindset: the thematic map, as well as all marks on it, are rather symbols than accurate representations. If this mindset is used while interpreting a thematic map, truthfulness and accuracy of a map can be defined as still showing the essence of geographic patterns and relationships \iacite{Tyner2010}.

The distinctness of map accuracy and map detail is important. The first generalization method called selection already mentions this implicit: accuracy refers to the amount of categories shown on the map, e.g. roads, rivers, lakes etc., whereas detail relates to the amount of information shown for each category, e.g. only show rivers with a certain size \iacite{Tyner2010}.

