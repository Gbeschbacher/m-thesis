The origin of cartography lays far back in the history of visualization as shown in chapter \ref{s:history} on page \pageref{s:history}. Today's understanding of modern cartography began in the late 18\textsuperscript{th} century with the attempt to show more than one attribute in a map. \citeauthor{Longley2005} say, that topography has long been understood as an important aspect of war strategy. Battles and wars could be decided on the information of the topology. A commander had to position his units wisely in order to exploit all geographical circumstances and thus have an advantage over his opponent \iacite{Longley2005}.
in the middle of the 20\textsuperscript{th} century, the Soviets, Fascist Italy and Nazi Germany all used map to foster national pride. This could furthermore be used to justify their expansionism \iacite{Crampton2015}.
Today, national organisations produce a variety of maps for different reasons with a very high focus of map accuracy (see chapter \ref{s:map-accuracy} on page \pageref{s:map-accuracy} for more information) and transmittal of relvant information.

The field of cartography can be divided into two major subcategories:
\begin{enumerate}
\ditem{General cartography} is associated with maps that are constructed for commonalty. These type of maps often contain a variety of features and display many reference and location systems. An abstract definition of general maps is that those kind of maps show the variety of phenomena of either geological, geographical or policital nature together \iacite{Thrower2008}.

\ditem{Thematic cartograpgy} focuses on a specific subject area, often called theme. Thus it involves maps which emphasize spatial variation of geographic distributions. \citeauthor{BartzPetchenik1979} describes thematic maps compared to general (or reference) maps as "in place, about space". She says, that the a general map is characterized by the fact that it shows you where something is in space, according to the example. However, thematic maps will tell a story about that specific place \iacite{BartzPetchenik1979}.
In order to make a connection to the abstract definition \citeauthor{Thrower2008} made for general maps, thematic maps could be described as follows:
Those kind of maps (thematic) use the base data only as points of references. Base data could be coastlines, boundaries and places. Phenomena of all kind are being mapped onto the reference \iacite{Thrower2008}.
\end{enumerate}

The following sections in this chapter should provide general knowledge about different types of maps, scaling, projecting, generalization and symbolization and accuracy. This knowledge is needed in order to understand the implementation of the partical part of this thesis.

\subsubsection{Types of thematic maps}
\subsubsection{Map Scale}
\subsubsection{Map Projections}
\subsubsection{Map Generalization}
\subsubsection{Symbolization}
\subsubsection{Map Accuracy}
\label{s:map-accuracy}
