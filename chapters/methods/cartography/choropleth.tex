For continuous data, two mapping techniques are commonly used: choropleth and isarithmic mapping. This chapter will only cover choropleth mapping because the practical part of this thesis will not feature isarithmic mapping. The basic idea of choropleth mapping is applying value or color intensity to enumeration units like census tracts, counties, states or nations based on some statistical value. The higher a value assigned to an enumeration unit, the more saturated the color of that unit. Fundamental to this type of mapping is standardization and classification of data. If data is not standardized and used raw, the visualization will suffer from an inherent areal bias \iacite{McMaster2010}. This problem is described with a practical example: the united kingdom has greater population but lower area compared to canada. However, if the mapped data is not standardized by area, canada will have a higher visual impact than united kingdom due to their superior areal extent, even though the mapped attribute mapped onto canada has a lower value compared to the one on united kingdom. Therefore the less the mapped attribute is tied to enumeration units, the less sense a choropleth map makes.

The following list shows the most rudimentary and most common methods to classify data mentioned by \citeauthor{McMaster2010}. The amount of classes must be known before and to explain the classifications below, five different classes with a value range from 10 to 85 is used.

\begin{itemize}

\item \textbf{Equal interval classification} assumes equal range between the class breaks. With the mentioned example, class one would include values from 10 to 25, class two would include values from 25 to 40, and so forth \iacite{McMaster2010}.

\item \textbf{Quantiles classification} needs to know the amount of items in a dataset in addition to the amount of desired classes. Consider 100 observations in a dataset and five desired classes. Thus, 20 observations will be placed in each class. These 20 observations can be thought of as data values. Putting this amount of data values in five different classes results in 4 data values per class. Now it is only needed to go through the whole dataset and put the first to the fourth item in the first class, the fifth to the ninth item in the second class, and so forth \iacite{McMaster2010}.

\item \textbf{Natural breaks classification} follows the idea of minimizing the internal variation of the dataset, while maximizing the variation among the classes. The user needs to choose significant gaps in the dataset, according to the number of desired classes \iacite{McMaster2010}.

\end{itemize}
