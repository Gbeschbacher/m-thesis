A map's accuracy should mainly be determined by its purpose. It is usually taken to mean positional or locational accuracy. A map, which is strictly positional accurate, means that all features displayed have the correct scale and are precise in their location. However, thematic maps are often drawn at a very small scale, making it impossible to achieve such an accuracy. Generalization impacts accuracy heavily. If generalization with a focus on certain aspects of a map is applied, the map's accuracy will suffer \iacite{Tyner2010}.

According to \citeauthor{Tyner2010}, it is still possible to be rather truthful and accurate. This can be achieved by establishing following mindset: the thematic map, as well as all marks on it, are rather symbols than accurate representations. If this mindset is used while interpreting a thematic map, truthfulness and accuracy of a map can be defined as still showing the essence of geographic patterns and relationships \iacite{Tyner2010}.

The distinctness of map accuracy and map detail is important. The first generalization method called selection already mentions this implicit: accuracy refers to the amount of categories shown on the map, e.g. roads, rivers, lakes etc., whereas detail relates to the amount of information shown for each category, e.g. only show rivers with a certain size \iacite{Tyner2010}.
