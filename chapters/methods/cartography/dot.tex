The first dot map in history is shown in figure \ref{fig:cholera-map} on page \pageref{fig:cholera-map}. It was the first map of its kind and could help in the display of disease outbreaks. This type of univariate thematic map uses points or dots to map discrete data. Attribute values of the given data determine the number of dots displayed in a specific regions. All dots need to be the same size. To explain the two different types of dot maps, imagine a dataset of customers where each customer has a location:

\begin{enumerate}
\ditem{One-to-one} \hfill \\
Each dot on the map represents exactly one item of the represented theme. Considering the example dataset mentioned above, every customer would be represented by exactly one dot.
\ditem{One-to-many} \hfill \\
Each dot on the map represents an aggregate of information. Therefore this type of dot map could be used with the aggregation of customers in a specific location. Thus the map maker needs to make the decision how many customers are aggregated, or rather how many customers are represented by one dot.
\end{enumerate}

Both types of dot density maps share the purpose, that they are not a tool to determine exact quantities. Getting the exact amount of dots in a high density area is a very cumbersome task and users often tend to underestimate dot totals as density increases \iacite{McMaster2001}. However, it is a very common technique for viewing the clustering, dispersion, linearity, and general pattern of a distribution. The technique appeared first in the 19\textsuperscript{th} century and is today accepted as one of the primary techniques for representing geographic patterns \iacite{Tyner2010}.

The map maker can use dots in a dot density map with a different type of level of detail. This means, that dots do not necessarily need to have an exact location. If he or she wants to discover a pattern on a state-wide level of detail, dots can be placed anywhere in their corresponding states, as long as they do not leave their state boundaries.
Another location based decision the map maker needs to make is, if the dots should use some kind of pseudo-random placement in case of overlapping. This decision is based on a maximum overlap constraint. It can be thought of as a random placement of dots in a square without violating the constraint.

According to \citeauthor{Tyner2010}, there are some main design principles for dot maps that should be considered:
\begin{itemize}
\item The size of the dot.
\item The value assigned to the dot. This also includes the correct use of the two different types of dot maps.
\item The location of the dot on the map in case of an aggregated level of detail of the map.
\item The aggregated units in case of a one-to-many dot map. This design principle can be thought of as using a legend in order to tell the aggregated value one dot represents.
\end{itemize}
Changing any one of these can change the overall appearance and interpretation of the map \iacite{Tyner2010}.

The main advantage of this type of map is the understandability. It requires little to no cognitive effort by the user to read the map when compared to other types. Specific advantages of dot maps are the good measure of density and the loose coupling between the size of a dot and its represented value.
However, reading specific information from those maps is not an easy task as mentioned before. Additionally, if a map uses some kind of random placement without any hint in the visualization, map readers may potentially infer the locations of dots as precise locations of the mapped phenomenon. To counteract the second drawback, dot density maps with random placement of dots should consider the acual occurence of the mapped phenomenon, e.g. dots should not be placed in lakes for a map of population.