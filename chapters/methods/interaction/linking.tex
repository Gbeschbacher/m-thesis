Multiple view systems are defined by using several views on one or more datasets, but with different aspects of it for each visualisation. Practical applications of all kinds make use of such a system, e.g. \ac{CAD} and \ac{GIS}. Even though the possibility of using multiple views on the same dataset is trivial, the implementation and necessary amount of interaction is not \iacite{Kosara2003}.

\citeauthor{Baldonado2000} present four design rules whether or not multiple view systems are appropriate for the specific task \iacite{Baldonado2000}:

\begin{enumerate}

\ditem{Diversity} \hfill \\
If the given dataset consists of attributes with different types, multiple levels of abstraction, and so forth, a multiple view system can be created. A single view would be overloaded with the given dataset because of the significant cognitive overhead created. The user would need to simultaneously comprehend and assimilate a multitude of diverse data.

\ditem{Complementarity} \hfill \\
Multiple view systems support visual comparability. They should be used if showing correlations and/or disparities is important, because they leverage multiple perceptual capabilities to improve understanding of relations among views.

\ditem{Decomposition} \hfill \\
Showing different attributes at the same time provides insight in different dimensions. This rule is related to the "divide and conquer" principle: the amount of data a user needs to consider at one time is reduced, thus aiding memory.

\ditem{Parsimony} \hfill \\
Multiple views build upon the cost of switching context and increasing complexity. The learning cost of a user, aswell as computational and display space costs of several views must be justifiable.

\end{enumerate}

Multiple view systems are often used for focus and zoom, e.g. for showing the focus and contex in seperate windows. \citeauthor{Robert:1998} describe a system based on this concept \iacite{Robert:1998}.

\citeauthor{Martin:1995} describe brushing as the process of selecting specific data items or groups of items to highlight them \iacite{Martin:1995}. Usually this process is initiated by directly interacting with the view using the mouse. Two possible manual interactions are opening a rectangular region of interest or clicking on a specific data items. It can also be accomplished without interacting with the visualisation itself by using sliders, select-boxes or even complex means like selecting a cluster \iacite{Kosara2003}.

However, using the brushing technique is not sufficient when exploring complex data efficiently. Linking describes the concept of exchanging information about which points are brushed. This concept is essential when using brushing and multiple view systems together. With this combination, a user can easily see the same points brushed in different views on the data \iacite{Kosara2003}.

