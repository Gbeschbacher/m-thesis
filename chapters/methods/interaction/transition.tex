In order to understand transition referring to visualisations, it is important to point out the difference between transition and animation. Even though these two terms are often substituting each other, they follow two different concepts \iacite{Muehlenhaus2014}:

According to the Merriam-Webster Online Dictionary\footnote{See \href{http://www.merriam-webster.com/}{Merriam-Webster Dictionary}} an animation is a way of making a movie by using a series of drawings, computer graphics or photographs that are slightly different from one another and that when viewed quickly one after another create the appearance of movement. Transition, however, is a movement, development, or evolution from one form, stage, or style to another.

With these definitions, it is easy to distinguish the two terms. Animation is the process of making any kind of movement visible to a user, whereas transitions in visualisations make use of animations to show e.g. the change of visual appearance.

\citeauthor{Thrower1959} was the first one to combine the research fields of cartography with animation. He describes the process of bringing e.g. population growth with animation on a map \iacite{Thrower1959}. However, animated cartography remained a bit of an oddity that was experimented with. This fact could not be changed even with the rise of the household \ac{PC}. Nonetheless, the mass adoption of the internet could counteract the rare usage of map animation \iacite{Muehlenhaus2014}.

This chapter furthermore describes considerations when designing animated maps. Animation can basically be broken down into two broad types:

\begin{enumerate}
\ditem{Stop-Frame Animation} \hfill \\
This type of animation is also known as stop motion. Every frame of this type of animation is designed separately. After many pictures and movements are made, all pictures are in the order they were taken. The rate of change for each picture heavily influences the cognition of the animation \iacite{Faroudja1991}.
\ditem{Tweening} \hfill \\
    This term bears its name from the word "betweening". It describes the process of interpolating movement between key frames, thus creating smoother animations compared to the stop-frame method \iacite{Muehlenhaus2014}.
\end{enumerate}

\citeauthor{DiBiase1992} identified three new visual variables dealing specifically with animation \iacite{DiBiase1992}:

\begin{enumerate}

\ditem{Duration} defines the temporal length of how long a particular frame in an animation is shown. Imagine a dataset containing decennial census of population data and an animation set to 24 frames per second. To give each map user the ability to conceive each decade, it is needed to show it at least for some seconds. For the sake of convenience, a duration of 2 seconds is assumed to be enough. Therefore, it is needed to show each decennial census data for 48 frames, resulting in a 2 second duration \iacite{Muehlenhaus2014}.

\ditem{Rate of Change} represents how quickly an image is morphed into the next attribute. In general, it denotes the magnitude of an attribute divided by the duration. This basically means that it is needed to know how much the animated image being represented changes from frame to frame \iacite{Muehlenhaus2014}.

\ditem{Order} denotes the sequential usage of data items. Rather than animating data in the order they are given, they can be reorganized first by a specific attribute. For example, one could show all countries with a high population first, before showing small ones \iacite{Muehlenhaus2014}.

\end{enumerate}

\citeauthor{Muehlenhaus2014} also mentions different types of map animations. He says weather forecasts are the most common use case for map animations. Deriving the term temporal animation for this kind of animation seems appropriate. Albeit showing change over time seems trivial, people do not always interpret the resulting animation accurately. He suggests the use of temporal animations for showing broad patterns of change in a dataset. However, the animation should either show broad changes in a small scale or detailed changes in a large scale. This heavily impacts the user's perception and interpretation of the animation, including "change blindness". This means, that obvious changes in a map are missed due to focusing something else in the map. In order to combat the problem of change blindness, \citeauthor{Muehlenhaus2014} names three different design principles \iacite{Muehlenhaus2014}:

\begin{enumerate}

\item \textbf{Animation duration} should be kept short in order to not overwhelm a user's short-term memory. Temporal animations should not take longer than 30 seconds to a minute, especially when showing changes in data, which are not narrative in nature. If this principle is ignored, data retention will be nullified \iacite{Muehlenhaus2014}.

\item \textbf{Data simplification} denotes the amount of animated attributes used. Showing more than four attributes in an animation results in visual overstimulation \iacite{Ware2008}.

\item \textbf{Control} needs to be given to the user. This can be achieved in many ways, e.g. by providing a pause or play button for an animation \iacite{Muehlenhaus2014}.

\end{enumerate}

Another type of animation \citeauthor{Muehlenhaus2014} mentions is the so called zoom animation. This type basically follows the main visual information seeking mantra of \citeauthor{Shneiderman1996}. If the purpose of an animation is to show a specific part of the data, it should start off by giving an overview. The animation is either started programmatically or manually and animates the focus to a specific part of the data. Control can ge given to the user in the form of freely moving around in the map or providing zoom buttons.

However, the mentioned types of animations are only based on a single visualisation. This thesis will use animation in a transition of one type of map visualisation to another. First, this should help to understand how the map is created. Second, some map visualisations are based on aggregations and therefore, the animation should show if this aggregation is understandable and interpretable.