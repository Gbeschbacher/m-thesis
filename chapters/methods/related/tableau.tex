Tableau Software is an american computer software company producing a variety of data visualization products focused on business intelligence. It offers five main products:
\begin{enumerate}
\ditem{Tableau Desktop} is used for all kinds of development. This tool is capable of creating a variety of visualizations and integrating interactions in all kinds of way depending on the visualization.

\ditem{Tableau Server} handles shared visualizations created in Tableau Desktop.

\ditem{Tableau Online} is the same as Tableau Server except a cloud-based solution.

\ditem{Tableau Reader} allows to open shared visualizations.

\ditem{Tableau Public} combines the core components from Tableau Desktop and Tableau Server.

\end{enumerate}

The rest of this section is related to the Tableau Desktop product, because it is the most interesting one. Its mission is to help people see and understand their data \iacite{Murray2013}.
Tableau offers data connections from databases, aswell as simple text-files and therefore can handle most kinds of dataset types. It even provides a web data connector allowing to create a connection to any kind of data accessible over HTTP. This can include internal web services, JSON data, REST APIs, and many other sources. Thus Tableau can handle static, aswell as dynamic datasets.
As \citeauthor{Murray2013} already stated, Tableau wants to simplify the process of understandability and interpretability. Therefore, customers of tableau consume their information and derive knowledge. This can be done by creating multiple linked visualizations with appropriate interaction. Tableau offers concepts like detail on demand, linking and brushing, creating temporal animations, and so forth.
The most important part of Tableau Desktop is it's dashboard. It is completly customizable, offering multiple methods to manipulate the given visualization with selections, visualization navigations and changes in visual appearances.

