Gapminder\footnote{See \href{https://www.gapminder.org/}{Gapminder online}} is very well known in the information visualization community and the content of the data it uses is entirely explored. It shows wealth and health of countries over time and has more than 500 datasets as a basis. The default visualization of gapminder is an extended scatter plot where the x-axis shows income per person, the y-axis life expectancy in years, the bubble itself represents the total population of each country by size and the color indicates the region of the country. Furthermore the scatter plot is expanded with a timeline. The interactive demo of Gapminder also allows users to inspect different indicators of wealth and health.

\begin{figure}[!htb]
\centering
\includegraphics[height=5cm]{images/methods/related/gapminder.png}
\caption[
    Gapminder
]{Gapminder}
\label{fig:gapminder}
\end{figure}

Gapminder is not visualization tool, but demonstrates a very popular example of animation in visualizations as a practical example. Figure \ref{fig:gapminder} on page \pageref{fig:gapminder} shows the application displaying the data corresponding to the year 1904. Users have some control over the visualization shown:
\begin{itemize}
\item changing the attributes mapped,
\item controlling the animation by playing and pausing it directly,
\item selecting the countries which are of interest and
\item setting the symbol size.
\end{itemize}

The application made it possible to discover that people live longer in countries with a higher \ac{GDP} per capita. Countries with a low income have really short life expectancy. Furthermore, it is also discovered, that living in middle income countries, the lifespan is huge, depending on how the income in the country is distributed and used.
