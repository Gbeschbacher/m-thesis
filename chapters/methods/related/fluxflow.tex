\citeauthor{Zhao2014} present an interactive visual analysis system designed for revealing and analyzing anomalies spreading in social media. The main challenge of such a system is to distill valuable information from streamed social media data, due to the heterogeneous and dynamic crowd behaviours \iacite{Zhao2014}. The challenge is not further described in this thesis because it is irrelevant as related work. However, some parts of the visual analysis system are very interesting and worth a rough analysis.

Figure \ref{fig:fluxflow} on page \pageref{fig:fluxflow} shows FluxFlow. It is based on multiple views combined with linking and brushing. As one can see, it deconstructs heterogeneous social media data and builds some kind of tree. The tree can be navigated and single nodes can be highlighted with hovering. Clicking on a single node creates a new view showing all its subnodes as circles on a timeline. The radius of the circle depends on an attribute of the data item. The creation of the timeline is linear animated. Each click on a node appends a new timeline in given view.

\begin{figure}[!htb]
\centering
\includegraphics[height=5cm]{images/methods/related/fluxflow.png}
\caption[
    Fluxflow: a visual analysis system \iacite{Zhao2014}.
]{Fluxflow: a visual analysis system}
\label{fig:fluxflow}
\end{figure}

The use case of such a system is well described in the paper, therefore making it easy to analyse possible datasets. Even though the main challenge of the system is to analyse streamed data, the visualisation itself is only based on static datasets. It is not clear if the system is able to build the tree from a table dataset itself or if the tree must be given.
\cbstart
The objective of FluxFlow is to show anomalies in social media data and discover insights with using the multiple view system combined with multiple visual channels and interaction methods. Manipulating the visualisation is possible through interaction with the timeline, selecting interesting nodes in the tree or entering specific time slots to view. Selecting a node in the tree results in a timeline where each children node is placed accordingly. The creation of the timeline is based on a fading animation. If there are aleady some nodes expanded as timelines, the axis is also animated and adjusted accordingly. The cruical point of the animation is that they are not shown simultaneously. These stacked transitions make it easier to follow the overall events in the application and a user can fully focus on each transition.
\cbend