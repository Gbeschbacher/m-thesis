At the beginning of this chapter, it is mentioned that the requirement of an interactive web application is given. However, this requirement needs to distinguish two different concepts of web applications:

\begin{enumerate}

\ditem{Pure client-side web application} \hfill \\
A client typically is a computer application, that runs on a user's local computer. With the context of a web application, a client is basically a web browser. In order to reach a web application, a client needs to connect to a server. With a pure client-side concept, the server returns static files only, thus leaving all further operations and calculations with the application on the client. The main advantage of such a concept is, that server costs can be significantly decreased and latency is basically not given when the first connection has been established. However, using the client for all further operations can be either an advantage or disadvantage, depending on the client's performance. If the client's hardware can handle a lot of operations, the web application will be extremely fast. However, if this is not the case and calculations, animations, etc. will take some time to be done, it will feel unnatural and unrewarding.

\ditem{Client-side web application combined with a server} \hfill \\
SandDance uses such a concept for its unit-based visualisation tool. The main difference to pure client-side web applications is, that operations and calculations are done on the server and the results are forwarded to the client. The client is only used to display information with animations and other concepts. This concept of a web-application needs a more abstract structure because client-server communication is obligatory. The information exchange will not affect a client in any way if the latency between a client and a server can be kept really low and the server has enough resources to perform fast.
\end{enumerate}

The technology used heavily depends on the decision of the type of web application. For the sake of convenience and no costs, the practical approach is implemented as a pure client-side application. Thus, the technological possibilities are restricted to \ac{SVG}, Canvas\footnote{See \href{https://developer.mozilla.org/de/docs/Web/HTML/Canvas}{https://developer.mozilla.org/de/docs/Web/HTML/Canvas} for more information.} and WebGL\footnote{See \href{https://developer.mozilla.org/de/docs/Web/API/WebGL_API}{https://developer.mozilla.org/de/docs/Web/API/WebGL\_API} for more information.}. The JavaScript library D3.js\footnote{See \href{https://d3js.org/}{https://d3js.org/} for more information.} makes it easy to create a \ac{SVG} in the browser. The creation can furthermore be extended with interactions. Canvas features a lot of 2D functionalities to create graphics and bitmap pictures. WebGL extends functionalities of canvas in 3D.