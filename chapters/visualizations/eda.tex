\citeauthor{Tukey1977} introduced and promoted the term of \ac{EDA} in 1977 to encourage statisticians to explore data and possibly formulate hypotheses that could lead to new data collection and experiments. Many \ac{EDA} ideas can be traced back to earlier authors, but \citeauthor{Tukey1977} summarized all of those ideas with the introduced term. Even though the earlier ideas have all been in combination with statistics, \citeauthor{Tukey1977} encouraged the research field of statistics to use the capabilities of dynamic visualisations in combination with given data. Before that encouragement, statisticians had a strong focus on statistical hypothesis testing without any visualisation. This development supported the identification of outliers, trends and patterns in data in a visual and understandable way \iacite{Tukey1977}. Thus it is possible to derive objectives of \ac{EDA} as follows \iacite{Behrens1997}:

\begin{itemize}

\item Suggest hypotheses about the causes of observed phenomena.
\item Assess assumptions on which statistical inference will be based
\item Support the selection of appropriate statistical tools and techniques
\item Provide a basis for further data collection through surveys or experiments.

\end{itemize}

As of today, there are a number of techniques assisting the field of \ac{EDA}, but the field is more characterized by the attitude taken than by particular techniques \iacite{Tukey1980}. Typical graphical techniques used in \ac{EDA} are box plots, histograms, scatter plots, stem-and-leaf plots, and so forth. This chapter will only discuss histograms in detail.

\citeauthor{Pearson1895} introduced histograms as a graphical representation of the distribution of numerical data. It is an estimate of the probability distribution of a quantitative variable \iacite{Pearson1895}. As the history of visualisation shows (see chapter \ref{s:history} on page \pageref{s:history} for more detail), cartography also had the goal to visualise distributions on thematic maps. It emphasized especially a spatial variation of one or a small number of geographic distributions.

With this in mind, this chapter could be taken to subsume the two main foci of the thesis: statistical graphics and thematic cartography. Both of these are concerned with graphical representations of quantitative and categorical data but driven by different representational goals. \citeauthor{Friendly.2001} describes the objectives of cartographic visualisation as "finding the representation constrained to a spatial domain" whereas statistical graphics try to apply to any domain in the service of statistical analysis \iacite{Friendly.2001}. In addition, cartography and statistical graphics share the common goals of visual representation for exploration and discovery. These range from the simple mapping of locations (land mass, rivers, terrain), to spatial distributions of geographic characteristics (species, disease, ecosystems), to the wide variety of graphic methods used to portray patterns, trends, and indications \iacite{Friendly.2001}. The combination of analytical reasoning with visual representations lead to the research field of visual analytics.

