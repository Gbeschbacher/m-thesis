\cbstart
The main goal of this Section is giving an overview of the definitions in literature of the term \textit{visualisation}. The first definitions appeared are taken from the beginnings of the research field of visualisation. The latter ones bear a reference to the technology available nowadays.
\cbend
Visualisation is first mentioned but not defined in 1953 in cartographic literature, in an article by University of Chicago geographer Allen K. Philbrick. \citeauthor{mccormick:1987} took up the term and defined it as follows:
\begin{quote}
 ``Visualisation is a method of computing. It transforms the symbolic into the geometric, enabling researchers to observe their simulations and computations. visualisation offers a method for seeing the unseen. It enriches the process of scientific discovery and fosters profound and unexpected insights. In many fields, it is already revolutionizing the way scientists do science.

 Visualisation embraces both image understanding and image synthesis. That is, visualisation is a tool for both for interpreting image data fed into a computer, and for generating images from complex multi-dimensional data sets. It studies those mechanisms in humans and computers which allow them in concert to preceive, use and communicate visual information \iacite{mccormick:1987}.''
\end{quote}

In 1987, new developments in the field of computer science prompted the National Science Foundation to redefine the term. The report of the redefinition placed visualisation at the convergence of computer graphics, image processing, computer vision, computer-aided design, signal processing, and user interface studies \iacite{mccormick:1987}. \citeauthor{Phillips2010} discuss three elemental issues existing today in the research field of visualisation:

\begin{enumerate}
\item to settle on a definition for the term \textit{visualisation}
\item to clarify the underlying presumptions and
\item to decide how to document both short-term and long-term effectiveness.
\end{enumerate}

\begin{quote}
The status of terms, often used interchangeably, such as \textit{visualisation}, \textit{visual representation}, \textit{visual media}, \textit{media literacy}, \textit{visual communication skills}, \textit{visual literacy}, \textit{illustrations}, and \textit{media illustrations}, is yet to be clarified. Furthermore, the routine confusion between pictures or visual images and reality is a fundamental and persistent problem \iacite{Phillips2010}.
\end{quote}

Because of the wide reach of the term, \citeauthor{Phillips2010} use a series of five steps in order to tell how \textit{visualisation} is defined in literature. The steps can be summarized as follows:
\begin{enumerate}
\item A search of all relevant sources, the identification of vocabulary and the mapping of the citations.
\item Classify the types of research into explanatory, exploratory, descriptive studies and ``other''.
\item Analyse and evaluate the assertions made in step two.
\item Organization of the reviews through comparisons of the literature, in order to identify areas of difference and similarity.
\item Mapping the collected information on several categories.
\end{enumerate}
The evaluation of this method used a total of 247 articles, ranging from the year 1936 to 2009. Out of those articles, 56\% were empirical studies and 44\% discussion articles. \citeauthor{Phillips2010} found out, that the attempt to define \textit{visualisation} in literature is not possible. The term is frequently substituted with terms like \textit{visual aid}, \textit{image}, \textit{visual literacy} etc.

To clarify the term with a relation to the progress made the last years, two well known online dictionaries are used. One possible generic definition could be the one from Oxford Dictionaries\footnote{See \href{https://www.oxforddictionaries.com/definition/english/visualisation}{https://www.oxforddictionaries.com/definition/english/visualisation}}:

\begin{quote}
\begin{enumerate}
\item ``The representation of an object, situation, or set of information as a chart or other image.''
\item ``The formation of a mental image of something.''
\end{enumerate}
\end{quote}

Even though the noun \textit{visualisation} has a very close related verb, \citeauthor{Phillips2010} noted the important distinction between those. The noun ``[\ldots] directs the attention to the product, the object, the 'what' of visualisation, the visual images. The verb of visualisation, on the other hand, makes one attend to the process, the activity, the skill, the 'how' of visualizing.''\iacite{Phillips2010}.

Another possible generic definition for \textit{visualisation} is taken from the Merriam Webster Online Dictionary\footnote{See \href{http://www.merriam-webster.com/dictionary/visualization}{http://www.merriam-webster.com/dictionary/visualization}}. It is close to the one taken from the Oxford Dictionaries but has one significant distinction: ``The act or process of interperting in visual terms or of putting into visible form''. Compared to the statements of \citeauthor{Phillips2010}, this dictionary does not distinguish the process and the act of \textit{visualisation}.

However, combining the already mentioned definitions of terms with the research results from \citeauthor{Phillips2010}, a three-fold distinction of definitions is visible:
\begin{enumerate}
\item Physical objects serving as visualisations (e.g. geometrical illustrations). These objects are viewed and interpreted by a person for the purpose of understanding something.
\item Mental objects pictured in the mind (e.g. mental imagery). These are imaginative constructions of some possible visual experiences.
\item Cognitive processing in which visualisations are interpreted, either physical or mental (e.g. cognitive manipulation of visual representations by the mind). This often refers to an act of deriving meaning from a physical or mental object.
\end{enumerate}

\cbstart
\citeauthor{Munzner2014} takes the available technology into account and therefore provides a more advanced definition. She combines the term \textit{visualisations} with computer-based systems yielding new definition possibilities to the whole term. Nonetheless, she does not revoke the older definitions and defines the term on the basis of them.
Using computer-based systems help people to carry out tasks more effectively.
\textit{Visualisations} created by systems cannot replace a human with computational decision-making methods, they are needed when there is a need to augment human capabilities. Furthermore, besides taking the process of creation into account, \textit{visualisations} also need to consider interaction with visual representations \iacite{Munzner2014}.

According to \citeauthor{Phillips2010} the distinction between physical and mental visualisation objects are obvious. This statement is also verified by the given definition of \citeauthor{mccormick:1987} where this exact distinction is already made. However the distinction between the visualisation itself and the thinking involved in interpreting it is also important \iacite{Phillips2010}. \citeauthor{Munzner2014} extends the term of \textit{visualisation} with the combination of computer-based systems and the thereby existing possibilities of interaction yielding a much broader term with three more limitations:
\begin{enumerate*}[label={(\arabic*)}]
\item the limitations of computers,
\item of humans,
\item and of displays \iacite{Munzner2014}.
\end{enumerate*}

\cbend