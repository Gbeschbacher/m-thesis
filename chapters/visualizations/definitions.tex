In order to fully understand this section, the definition of the term \textit{visualisation} needs to be clear (see Chapter \ref{s:definition} on page \pageref{s:definition} for more information).

So far the term of visualisation has been defined. Formerly, the research field of visualisation was often split in two major parts: the field of \ac{InfoVis} and \ac{SciVis}. The following list briefly describes each field. However, the distinction between those fields becomes more and more blurred. This is due to the fact that the data used in visualisations are not necessarily of a single nature, e.g. geo-spatial data only. With today's possibilities in data acquisition and preprocessing, there are no limitations in combining data from different natures.

\begin{description}

\item[\acl{InfoVis}] \hfill \\
\citeauthor{Friendly.2001} refered to \ac{InfoVis} to the broadest term that could be used to group all the developments mentioned in Section \ref{s:history}. From a very basic point of view, almost everything, if sufficiently organised, contains information in some kind. Therefore the term could be used for the earliest attempts to scratch information on rocks and to the earliest use of diagrams in history \iacite{Friendly.2001}.

\citeauthor{Ferreira2003} had a more contemporary definition of \ac{InfoVis}. They said the graphical models may represent abstract concepts and relationships that do not necessarily have a counterpart in the physical world. An example of the mentioned relationship would be an information describing user accesses to pages of an internet portal. It is typical for each data unit to describe multiple related attributes (usually more than four) that are not of some kind of spatial or temporal nature. Although spatial and temporal attributes may occur, the data existed in an abstract data space \iacite{Ferreira2003}.
\newpage
\item[\acl{SciVis}] \hfill \\
A close related, yet distinct field, called \ac{SciVis}, was primarily concerned with the visualisation of 3D phenomena, where the emphasis was on realistic renderings of volumes, surfaces, etc. 3D Phenomena for \ac{SciVis} could be of architectural, meteorological, medical nature and so forth \iacite{Friendly.2001}.

\citeauthor{Ferreira2003} also had a similar definition for \ac{SciVis}. They defined the term as the graphical models which are typically constructed from measured or simulated data representing objects or concepts associated with phenomena from the physical world. As such, the derived visualisations represented objects that exist in a 1D, 2D or 3D object space. The presence of spatial and temporal dimensions could be included in the data and thus are determinants in deriving visual representations from it \iacite{Ferreira2003}.
\end{description}

In the early 1980s, a french graphic theorist named Jacques Bertin set a milestone in the area of scientific research, as already mentioned in Chapter \ref{s:history} on page \pageref{crossref:bertain}. Based on his work, the Semiology of Graphics, the research field called \ac{GeoVis} arose. His work shows a strong focus in the research of the potential for the use of dynamic visual displays as prompts for scientific insight and of the methods through which dynamic visual displays might leverage perceptual and cognitive processes to facilitate scientific thinking \iacite{maceachren:2004}.

The broad term \ac{GeoVis} has the same problems as the other one concerning a specific definition: the term is used in many different research fields. Thus, multiple definitions suitable for the particular case exist in literature. This thesis gives an overview of some definitions, which also apply to the use of the term in this thesis.
The following definition according to the 2001 research agenda of \ac{ICA}, Commission on visualisation and Virtual Environments, is most widely accepted:
\begin{quote}
``Geovisualisation integrates approaches from visualisation in scientific computing, cartography, image analysis, information visualisation, exploratory data analysis and geographic information systems to provide theory, methods, and tools for visual exploration, analysis, synthesis, and presentation of geospatial data \iacite{Longley2005}.''
\end{quote}

\citeauthor{Noellenburg2007} mentions other definitions which have a more human-centred view and describe geovisualisation as
\begin{quote}
``[\ldots] the creation and use of visual representations to facilitate thinking, understanding, and knowledge construction about geospatial data'', or as ``the use of visual geospatial to display and to explore data and through that exploration to generate hypotheses, develop problem solutions and construct knowledge \iacite{Noellenburg2007}.''
\end{quote}

\newpage
Even though the definitions are slightly constrasting, they share one thought: \ac{GeoVis} is a multidisciplinary task and since it is a human being using those visualisations to explore data and construct knowledge, \ac{GeoVis} takes the user needs into account above everything else.