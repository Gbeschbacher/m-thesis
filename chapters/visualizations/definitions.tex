In order to fully understand this section, it is needed to know the definition of the term "visualization" (see chapter \ref{s:definition} on page \pageref{s:definition} for more information.).

So far the term of visualization has been defined. This is used as a basis to define the following terms: \ac{DataVis}, \ac{GeoVis}, \ac{InfoVis} and \ac{SciVis}. Even though there are a lot more practical applications for visualizations such as Educational visualization, Product visualization, only the four mentioned are explained in detail because they are important to understand the rest of this paper.

\begin{enumerate}

\ditem{\acl{InfoVis}} \hfill \\
\citeauthor{Friendly.2001} refer to \ac{InfoVis} to the broadest term that could be used to group all the developments mentioned in section \ref{s:history} on \pageref{s:history}. From a very basic point of view, almost everything, if sufficiently organized, contains information in some kind. Therefore the term can be used for the earliest attempts to scratch information on rocks and to the earliest use of diagrams in the history \iacite{Friendly.2001}.

\citeauthor{Ferreira2003} have a more contemporary definition of \ac{InfoVis}. They say the graphical models may represent abstract concepts and relationships that do not necessarily have a counterpart in the physical world. An example to the mentioned relationship would be a information describing user accesses to pages of an internet portal. It is typical for each data unit to describe multiple related attributes (usually more than four) that are not some kind of spatial or temporal nature. Although spatial and temporal attributes may occur, the data exists in an abstract data space \iacite{Ferreira2003}.

The later definition is closer to the term usage of today. It is generally applied to the visual representation of large-scale collections of non-numerical information.

\ditem{\acl{SciVis}} \hfill \\
A close related, yet distinc field, called \ac{SciVis}, is primarily concerned with the visualization of 3D phenomena, where the emphasis is on realistic renderings of volumes, surfaces, etc. 3D Phenomenas for \ac{SciVis} could be architectural, meterological, medical nature and so forth \iacite{Friendly.2001}.

\citeauthor{Ferreira2003} also have a similar definition for \ac{SciVis}. They define the term as the graphical models which are typically constructed from measured or simulated data representing objects or concepts associated with phenomena from the physical world. As such the derived visual representations represent objects that exist in a 1D, 2D or 3D object space. The presence of spatial and temporal dimensions could be included in the data and thus are determinant factors in deriving visual representations from it \iacite{Ferreira2003}.

\ditem{\acl{DataVis}} \hfill \\
This specific type of visualization has a slightly narrower domain as the others. It is best described as the science of visual representation of "data", defined as information which has been abstracted in some schematic form, including attributes or variables for the units of information \iacite{Friendly.2001}.
With the existing basis of the term, it is extended a little for this paper: \ac{DataVis} is not only the science of visual representation but also the process of using graphical presentation to represent data in a way that provides qualitative understanding of its information contents, turning complicated sets of data into visual insights.

\ditem{\acl{GeoVis}} \hfill \\
In the early 1980s, a french graphic theorist named Jacques Bertin set a milestone in the area of scientific research, as already mentioned in chapter \ref{s:history} on page \pageref{s:history}. Based on his work, the semiology of graphcics, the research field called \ac{GeoVis} developed. His work shows a strong focus in the research of the potential for the use of dynamic visual displays as prompts for scientific insight and of the methods through which dynamic visual displays might leverage perceptual cognitive processes to facilitate scientific thinking \iacite{maceachren:2004}.

The broad term \ac{GeoVis} has the same problems as the other one concerning a specific definition: the term is used in many different research fields, thus multiple definitions suitable for the specific case exist in literature. This paper gives an overview of some definitions, which also apply to the use of the term in this paper.
The following definition according to the 2001 research agenda of \ac{ICA} Comission on Visualization and Virtual Environments is most widely accepted:
\begin{quote}
"Geovisualization integrates approaches from visualization in scientific computing, cartography, image analysis, information visualization, exploratory data analysis and geographic information systems to provide theory, methods and tools for visual exploration, analysis, synthesis, and presentation of geospatial data \iacite{Longley2005}."
\end{quote}

\citeauthor{Noellenburg2007} mentions other definitions which have a more human-centered view and describe geovisualizatoin as
\begin{quote}
"the creation and use ofvisual representations to facilitate thinking, understanding, and knowledge construction about geospatial data", or as "the use of visual geospatial to display and to explore data and through that exploration to generate hypotheses, develop problem solutions and construct knowledge \iacite{Noellenburg2007}."
\end{quote}

Even though the definitions are slightly constrasting, they share one thought: \ac{GeoVis} is a multidisciplinary task and since it is a human being using those visualizations to explore data and construct knowledge, \ac{GeoVis} needs to take the user needs into account above everything else.
Based on this observation, the close trelated fields of \ac{GeoVis} are visible: the very basics of this type of visualization are still some kind of \ac{InfoVis}. Also \ac{SciVis} can be considered to be related in terms of emphasizing the knowledge creation and hypothesis generation aspects of \ac{SciVis}. The relation to \ac{DataVis} is not a stringent necessity because a lot of visualizations of \ac{DataVis} can be made without any spatial aspect. However, if any spatial attributes exist in the given dataset, it is most likely to require the use of maps or 3D displays where at least two display dimensions are utilised to represent the physical space \iacite{ICA2007}.
This is the contrary of \ac{InfoVis}, which only deals with abstract data spaces.
\end{enumerate}