\subsection{Conceptual Results}
\label{s:animated-transitions-implemented}
The implemented animated transitions between visual appearances are shown in Table \ref{tab:transition-table-implemented} on page \pageref{tab:transition-table-implemented}. Every transition is described with abbreviations in the table which are explained in detail in the description list below. A cell containing several abbreviations denotes the transition flow step by step. Thus, the list consists of basic transitions which are combined in various ways in order to achieve an animated transition from one visualisation type to another one.

\begin{description}
\item[Area-colour] \hfill \\
Colouring enumeration units where the colour either is determined by the current selected colour scheme or the default colour.

\item[CB-colour] \hfill \\
Centroid-based colouring is only used in conjunction the \textit{Draw default}-transition. Particles moving to its centroid colour their appropriate symbol when reaching the centroid.

\item[Draw defaults] \hfill \\
Default symbols (circles) are drawn for each enumeration unit at its centroid. Each circle has a default radius. This transition state is mainly used to set up following ones.

\item[Force] \hfill \\
This transition state is only used to create cartograms. Force is dependent on symbols in the screen. It applies collision detection and gravity on those symbols in order to create the Pseudo-Demers cartogram.

\item[Origin-colour] \hfill \\
Origin-colour is the exact opposite of "CB-Colour. All particles are moving from their centroid to their origin location. When leaving their appropriate symbol, the colour of the symbol adapts accordingly.

\item[Scale] \hfill \\
Scaling is also dependent on existing symbols. When scaling is triggered, all symbols are either scaled to their population or back to a default value.

\item[Symbol-colour] \hfill \\
All symbols are coloured at once appropriately to their amount of orders.

\item[Symbol-translation] \hfill \\
If the current visualisation is a cartogram and the user wishes to change the visual appearance, all symbols need to move back to their origin again.

\end{description}

\begin{table}[!htp]
    \begin{tabular}{M{27mm}|| M{27mm} | M{27mm} | M{27mm} | M{27mm} N}

    ~ & Dot Map & Proportional Symbol Map & Choropleth Map & Cartogram &\\[4ex] \hline \hline

    Dot Map & ~ & Scale, Origin-colour & Draw defaults, Area-colour, Symbol-colour, Origin-colour & Symbol-translation, Scale, Origin-colour &\\[4ex] \hline

    Proportional Symbol Map & Draw defaults, CB-colour, Scale & ~ & Draw defaults, Area-colour, Symbol-colour, Scale & Symbol-translation &\\[4ex] \hline

    Choropleth Map & Draw defaults, CB-colour, Area-colour & Scale, Area-colour & ~ & Symbol-translation, Scale, Area-colour &\\[4ex] \hline

    Cartogram & Draw defaults, CB-colour, Scale, Force & Force & Draw defaults, Area-colour, Symbol-colour, Scale, Force & ~ &\\[4ex]

    \end{tabular}
    \caption{Transition table showing the implemented transition from a given visualisation (column) to any upcoming visualisation (rows)}
    \label{tab:transition-table-implemented}
\end{table}

The implemented web application is accessible online\footnote{See \href{https://particles-masterthesis.github.io/aggregation/}{https://particles-masterthesis.github.io/aggregation/} for more information.}. The figures featured in this section are all taken directly from the web application. The source code is also accessible online\footnote{See \href{https://github.com/particles-masterthesis/aggregation}{https://github.com/particles-masterthesis/aggregation} for more information.}.

The implemented application only features basic interaction methods. Figure \ref{f:showcase-overall} on page \pageref{f:showcase-overall} shows a navigation bar and the configurable settings. Changing the visual appearance can be achieved by interacting with the bar. If changing any kind of configuration is desired, e.g. the speed of moving particles or the level of detail shown, this can be done by using the graphical user interface in the top right.

\begin{figure}[!htb]

   \begin{minipage}{\linewidth}
        \centering
        \subcaptionbox
        [
            Particle-based dot map
        ]
        {
            Particle-based dot map.
            \label{f:showcase-dot}
        }
        [.4\linewidth]
        {
            \includegraphics[width=0.4\textwidth,keepaspectratio]
            {images/results/map_dot.png}
        }
        \qquad
        \subcaptionbox
        [
          Multivariate classed proportional symbol map
        ]
        {
            Multivariate classed proportional symbol map
            \label{f:showcase-psm}
        }
        [.4\linewidth]
        {
            \includegraphics[width=0.4\textwidth,keepaspectratio]
            {images/results/map_psm.png}
        }
   \end{minipage}

   \begin{minipage}{\linewidth}
        \centering
        \subcaptionbox
        [
            Classed choropleth map.
        ]
        {
            Classed choropleth map.
            \label{f:showcase-choropleth}
        }
        [.4\linewidth]
        {
            \includegraphics[width=0.4\textwidth,keepaspectratio]
            {images/results/map_choropleth.png}
        }
        \qquad
        \subcaptionbox
        [
          Classed Pseudo-Demers cartogram.
        ]
        {
            Classed Pseudo-Demers cartogram
            \label{f:showcase-cartogram}
        }
        [.4\linewidth]
        {
            \includegraphics[width=0.4\textwidth,keepaspectratio]
            {images/results/map_cartogram.png}
        }
   \end{minipage}
    \caption[The implemented web-application showing four different types of geo-spatial map visualisations and with its interaction possibilities.]
    {The implemented web-application showing four different types of geo-spatial map visualisations and with its interaction possibilities.}
    \label{f:showcase-overall}
\end{figure}



Furthermore, Figure \ref{f:showcase-overall} on page \pageref{f:showcase-overall} shows the implemented geo-spatial map visualisations. Figure \ref{f:showcase-dot} shows the implemented dot map, \ref{f:showcase-psm} the multivariate classed proportional symbol map, \ref{f:showcase-choropleth} the classed choropleth map and \ref{f:showcase-cartogram} the classed Pseude-Demers cartogram.

One herein before mentioned animated transition is showcased in Figure \ref{f:showcase-transition-dot-cartogram} on page \pageref{f:showcase-transition-dot-cartogram}. It shows the animated transition from the dot map to the Pseudo-Demers cartogram. Table \ref{tab:transition-table-implemented} on page \pageref{tab:transition-table-implemented} explaines the stacked transitions in detail. The figure needs to be read line by line. The level of detail in the example is set to use state-data. Step 1 shows the dot map which is the starting point of the transition. Step 2 illustrates the default symbols (Draw defaults). Step 3 to 5 are showing the CB-colour transition where all particles are moving to their appropriate centroid and colouring it if they reach their destination. If this transition is finished, step 6 and 7 show the scaling of the symbols according to their population. Step 8 displays the force which is applied in case of a collision and the remaining subfigures showcase the location preserable gravity applied to each symbol. If the transition is entirely finished, the base map is removed.

\begin{figure*}
    \centering

    \begin{subfigure}[b]{0.31\textwidth}
        \centering
        \includegraphics[width=\textwidth]{images/results/dot_cartogram/transition_1.png}
        \caption[]%
        {{\small Step 1}}
    \end{subfigure}
    \hfill
    \begin{subfigure}[b]{0.31\textwidth}
        \centering
        \includegraphics[width=\textwidth]{images/results/dot_cartogram/transition_2.png}
        \caption[]%
        {{\small Step 2}}
    \end{subfigure}
    \hfill
    \begin{subfigure}[b]{0.31\textwidth}
        \centering
        \includegraphics[width=\textwidth]{images/results/dot_cartogram/transition_3.png}
        \caption[]%
        {{\small Step 3}}
    \end{subfigure}
    \vskip\baselineskip

    \begin{subfigure}[b]{0.31\textwidth}
        \centering
        \includegraphics[width=\textwidth]{images/results/dot_cartogram/transition_4.png}
        \caption[]%
        {{\small Step 4}}
    \end{subfigure}
    \hfill
    \begin{subfigure}[b]{0.31\textwidth}
        \centering
        \includegraphics[width=\textwidth]{images/results/dot_cartogram/transition_5.png}
        \caption[]%
        {{\small Step 5}}
    \end{subfigure}
    \hfill
    \begin{subfigure}[b]{0.31\textwidth}
        \centering
        \includegraphics[width=\textwidth]{images/results/dot_cartogram/transition_6.png}
        \caption[]%
        {{\small Step 6}}
    \end{subfigure}
    \vskip\baselineskip

    \begin{subfigure}[b]{0.31\textwidth}
        \centering
        \includegraphics[width=\textwidth]{images/results/dot_cartogram/transition_7.png}
        \caption[]%
        {{\small Step 7}}
    \end{subfigure}
    \hfill
    \begin{subfigure}[b]{0.31\textwidth}
        \centering
        \includegraphics[width=\textwidth]{images/results/dot_cartogram/transition_8.png}
        \caption[]%
        {{\small Step 8}}
    \end{subfigure}
    \hfill
    \begin{subfigure}[b]{0.31\textwidth}
        \centering
        \includegraphics[width=\textwidth]{images/results/dot_cartogram/transition_9.png}
        \caption[]%
        {{\small Step 9}}
    \end{subfigure}
    \vskip\baselineskip

    \begin{subfigure}[b]{0.31\textwidth}
        \centering
        \includegraphics[width=\textwidth]{images/results/dot_cartogram/transition_10.png}
        \caption[]%
        {{\small Step 10}}
    \end{subfigure}
    \hfill
    \begin{subfigure}[b]{0.31\textwidth}
        \centering
        \includegraphics[width=\textwidth]{images/results/dot_cartogram/transition_11.png}
        \caption[]%
        {{\small Step 11}}
    \end{subfigure}
    \hfill
    \begin{subfigure}[b]{0.31\textwidth}
        \centering
        \includegraphics[width=\textwidth]{images/results/dot_cartogram/transition_12.png}
    \end{subfigure}
    \vskip\baselineskip

    \caption[Showcase of the implemented animated transition from a dot map to the Pseudo-Demers cartogram]
    {\small Showcase of the implemented animated transition from a dot map to the Pseudo-Demers cartogram}
    \label{f:showcase-transition-dot-cartogram}
\end{figure*}


\subsection{Technical Details}
\cbstart
This Section will discuss all technical details according to the implementation.
As already mentioned earlier, the source code is available online\footnote{See \href{https://github.com/particles-masterthesis/aggregation}{https://github.com/particles-masterthesis/aggregation} for more information.}. All Listings discussed and shown in this Section are taken directly from the source code without any changes.
\cbend

\subsubsection{Requirements}
The pracitcal application needs to meet some basic conditions, which are described below:

\begin{description}
\item[Scaling:] unit-based visualizations allow for showing a lot of items simultaneously and combined with animation, some patterns in data can be revealed. However, using a low amount of units will not result in showing patterns animations. Therefore, usig a mid-sized dataset is a key part of the practical implementation. Defining a mid-sized dataset is nebulous and for this thesis, it is defined as a dataset consisting of 5000 to 15000 rows. This amount should be enough for an experiment to investigate the question of how an aggregation can be realized.

\item[Modularity:] the practical implementation should have a modular structure in order to keep it extendable. It should be possible to add further visualizations and transitions for more investigation later on.

\item[Client-side application:] in order to save costs and abstractions, a client-side approach is required. This allows to put the focus on the implementation of different thematic maps and an animated transition between them without worrying about latency of server response.

\end{description}

\subsubsection{Data Acquisition}
\label{s:data-acquisition}
Tableau uses a very interesting sample dataset to show basic features. It is called "Superstore-Sale" and contains approximately 10000 rows. Superstore-Sale features a lot of different attributes for each item. It comes close to the example of e-commerce mentioned in chapter \ref{s:geovis-practical} on page \pageref{s:geovis-practical}. The dataset mainly consists of information about office products, which have been bought in the period of time from 2011 to 2014. Furthermore, each order in the dataset includes some parts of the delivery address. This information about country, city, state and postal code allows to use the dataset with a map because it features some kind of geographical information. However, before being able to use the dataset on a map, it must be expanded by longitude and latitude first. This part is explained in detail in section \ref{s:data-preprocessing} on page \pageref{s:data-preprocessing}.

Having a dataset with interesting information leads to the second part of this section. A thematic map does not only consist of its symbols or coloured areas, but it also needs to show the geographical conditions in the form of a map. Superstore-Sale only includes items from the United States, thus a map of this location is needed. The U.S. Census Bureau publishes cartographic boundaries as shapefiles for thematic mapping. They provide multiple resolutions for different use cases. For the chosen dataset, the lowest resolution will show enough features in the map. Being able to see forests, streets, and so forth, in the map is not needed to show an overview of a commercial dataset. An advantage using a low resolution dataset is the file size. This must be considered, because all information needs to be delivered to the client, thus delivering 1 \ac{Mb} or 100\ac{Mb} of information is significant in terms of loading time of the web application.
Furthermore, the topicality of the cartographic boundaries needs to be considered. Usually, county boundaries do not change frequently, so it is acceptable to use the decennial census rather than the most recent release. Listing \ref{lst:data-acqu-zip} on page \pageref{lst:data-acqu-zip} shows an automated way of downloading the decennial version of the lowest resolution cartographic boundaries. It is accomplished with GNU-Make\footnote{See \href{https://www.gnu.org/software/make/}{https://www.gnu.org/software/make/} for more information.}. The target name without the directory, \textit{gz\_2010\_us\_050\_00\_20m}, implicitly has some information:

\begin{itemize}
\item \textbf{2010} is the release year of the file
\item \textbf{us\_050\_00} refers to boundaries of the United States (two characters abbreviation and the country id)
\item \textbf{20m} denotes the resolution (1:20.000.000)
\end{itemize}

%TC:ignore
\begin{lstlisting}[style={makefile}, caption={Make task for downloading cartographic boundaries}, label={lst:data-acqu-zip}]
/*build/gz_2010_us_050_00_20m.zip*/:
    mkdir -p $(dir $@)
    curl -o $@ http://www2.census.gov/geo/tiger/GENZ2010/$(notdir $@)
\end{lstlisting}
%TC:endignore

Furthermore, some thematic maps are going to use population as a mapped attribute in the map and therefore the acquisition of this data still needs to be shown. Listing \ref{lst:data-acqu-pop} on page \pageref{lst:data-acqu-pop} looks similar to the one already shown. The only difference is the source of the data.

%TC:ignore
\begin{lstlisting}[style={makefile}, caption={Make task for downloading population of counties}, label={lst:data-acqu-pop}]
/*build/population.zip*/:
    mkdir -p $(dir $@)
    curl -o $@ http://api.censusreporter.org/1.0/data/download/latest?table_ids=B01003&geo_ids=050|01000US&format=csv

\end{lstlisting}
%TC:endignore

Executing the mentioned tasks will download the zip-files from the sources and save them in the build directory.


\subsubsection{Data Preprocessing}
\label{s:data-preprocessing}
As the section before already showed, two different files are now accessible. However, in order to actually use these files, some preprocessing is needed. This chapter is divided into three parts. The first one will dicsuss the preparation of the Superstore-Sale dataset, the second one will explain how to get boundaries out of the downloaded zip-file and the third one will demonstrate how these two files can be merged.

\subsubsection{Preprocessing Superstore-Sale}
First of all, some numbers in this dataset are shown in a different format. Converting all numbers to the same format is essential.
Secondly, as already mentioned, the dataset only features attributes like country, city, state and postal code. In order to show the items on the map, it is necessary to expand each item with longitude and latitude. Listing \ref{lst:data-prep-latlong} on page \pageref{lst:data-prep-latlong} shows how the expansion can be accomplished when using NodeJs\footnote{See \href{https://nodejs.org/en/}{NodeJs} for more information.}. The listing uses Nominatim\footnote{See \href{https://nominatim.openstreetmap.org/}{Nominatim} for more information.} to decode geographical location strings to longitude and latitude. Furthermore listing \ref{lst:data-prep-latlong} shows, that each item is extended two more attributes: \textit{CountyId} and \textit{StateId}. These attributes denote the geographical id to the corresponding county and state name. As a starting point, a file containing all zip-codes with their corresponding ids was needed. Jgoodall provides such a file in his us-map repository\footnote{See \href{https://github.com/jgoodall/us-maps/}{his repository} for more information.}. Reading this file results in a lookup dictionary, where it is possible to search each postal code for its corresponding ids. However, Superstore-Sale does not provide a leading zero if a postal code is less than five characters, leading to a special case, which has to be considered.

\begin{lstlisting}[caption={Preprocessing Superstore-Sale with latitude and longitude}, label={lst:data-prep-latlong}]

let zipCodes = require("./zip-codes.json");
// Add latitude and longitude if there is a geo information
if (item.hasOwnProperty("Country") && item.hasOwnProperty("City") && item.hasOwnProperty("State")) {
    let url = `http://nominatim.openstreetmap.org/search?email=${config.email}&format=json&`;
    url += `country=${item.Country}&`;
    url += `state=${item.State}&`;
    url += `city=${item.City}`;

    const response = request("GET", url, {
        "headers": {
            "user-agent": "University of Applied Sciences Salzburg - Masterthesis Particles - MMT-M2014"
        }
    });
    const data = JSON.parse(response.getBody());
    item.Latitude = data[0].lat;
    item.Longitude = data[0].lon;
}

if (item.hasOwnProperty("Postal Code")) {
    let currentZip = item["Postal Code"];
    if(zipCodes.hasOwnProperty(currentZip)){
        let codes = zipCodes[currentZip];
        item.CountyId = codes.countyId[0];
        item.StateId = codes.stateId;
    } else {
        // superstore has no leading 0 of postal codes if code length < 5;
        currentZip = "0".concat(currentZip)
        if(zipCodes.hasOwnProperty(currentZip)){
            let codes = zipCodes[currentZip];
            item.CountyId = codes.countyId[0];
            item.StateId = codes.stateId;
        } else {
            console.log(`\${currentZip} not found!`);
        }
    }
}
\end{lstlisting}

\subsubsection{Preprocessing Cartographic Boundaries}


\subsubsection{Merging Superstore-Sale with Cartographic Boundaries}


\subsubsection{Used technologies}
The web-application is written in JavaScript ES6\footnote{See \href{http://www.ecma-international.org/ecma-262/6.0/}{http://www.ecma-international.org/ecma-262/6.0/} for more information.}. The development is done under Linux and the application is tested on Linux and Windows.

\paragraph{D3}
D3.js\footnote{See \href{https://d3js.org/}{https://d3js.org/} for more information.} is a library based on JavaScript. It allows to bind arbitrary data to a \ac{DOM}, and apply transformations to the document afterwards. \ac{D3} makes use of \ac{HTML}, \ac{SVG} and \ac{CSS}. Using these web standards offers full capabilities of modern browsers. It includes powerful visualisation components and a data-driven approach to \ac{DOM} manipulation. \ac{D3} is supposed to be really fast, supporting large datasets and dynamic behaviours for interaction and animation. The component-based architecture of \ac{D3} allows for code reuse.

The practical application uses \ac{D3} for two differnt things:
\begin{enumerate}
\item The base map is drawn using \ac{D3} because it features a lot of different map projections, allows to use the client-side library of TopoJSON, and overall has a lot of useful interactions already implemented.
\item All thematic maps based on aggregation are realised with \ac{D3} because of its use of \ac{SVG}. It provides functionality in e.g. calculating the centroid of a given location. Furthermore, \ac{D3} offers scheduled transitions with \ac{SVG}-objects making it a reasonable choice.
\end{enumerate}

\paragraph{PixiJS}
\ac{Pixi} is a fast and lightweight 2D library\footnote{See \href{http://www.pixijs.com/}{http://www.pixijs.com/} for more information.} built upon Canvas technology. Its renderer allows to enjoy e.g. hardware acceleration without the prior knowledge of \ac{WebGL}. Thus, \ac{Pixi} simplifies the creation of rich, interactive, cross platform applications without the need of knowledge of browser and device compatibility. Another characteristic of \ac{Pixi} is the seamless fallback of HTML5's canvas in case the browser is not supporting \ac{WebGL}.
\ac{Pixi} can therefore be used to draw a lot of items with the power of hardware acceleration, making it a good choice to draw a dot map with a lot of data items.

\paragraph{Noteworthy Development Extensions}
As already mentioned, the web-application uses the JavaScript version ECMAScript6. However, to use this version, a transpiler is needed. Babel\footnote{See \href{https://babeljs.io/}{https://babeljs.io/} for more information.} was chosen to fulfill this role because of its popularity. Furthermore, browserify\footnote{See \href{http://browserify.org/}{http://browserify.org/} for more information.} was used to maintain a component-based architecture and allowing to follow the concept of seperating concerns in files.
In order to keep an eye on performance, Stat.js\footnote{See \href{https://github.com/mrdoob/stats.js/}{https://github.com/mrdoob/stats.js/} for more information.} was used to display the current frames per second. It also allows to keep track of the milliseconds needed to render a frame and the \ac{Mb} of allocated memory. DatGui\footnote{See \href{https://github.com/dataarts/dat.gui}{https://github.com/dataarts/dat.gui} for more information.} simplifies the process of allowing the user to change configuration variables and therefore interacting with the visualisation.


\subsubsection{Web Application}
\label{s:web-application}
The first subsection will explain the system's architecture in detail based on a \ac{UML} diagram. With this diagram, it is possible to show the component-based architecture and how the mentioned tools are combined for the different visualisations and concepts. The second subsection will show the transition manager in detail, which features the transition table mentioned in Chapter \ref{s:theoretical-contrib} on page \pageref{s:theoretical-contrib}.
As already mentioned in Chapter \ref{s:collaboration-statement} on page \pageref{s:collaboration-statement}, \citeauthor{Wanko2016} is researching on a related topic of this thesis. The practical part of his thesis and the one of this thesis are implemented in the same system. Therefore, the actual application has a lot more features and options than described in this Section.

\subsubsection{Application Architecture}
Figure \ref{fig:uml-practical-approach} on page \pageref{fig:uml-practical-approach} gives an overview of the architecture in the form of a class diagram. However, the diagram shown and discussed in the master-thesis of \citeauthor{Wanko2016} looks quite different, although it is based on the same system. This is due to the relatively big system architecture and the different scopes of both theses \iacite{Wanko2016}. Showing and discussing all available classes, features and options for the application would go beyond the scope of this thesis. The following list will briefly discuss the most important classes and their purposes:

%TC:ignore
\begin{figure}[!htb]
\centering
\includegraphics[width=0.8\textwidth,keepaspectratio]{images/results/dia.png}
\caption[
    Overview of the application architecture in the form of a class diagram.
]{Overview of the application architecture in the form of a class diagram.}
\label{fig:uml-practical-approach}
\end{figure}
%TC:endignore

\begin{description}
\item[DataStore] \hfill \\
An object of the DataStore-class is used to load the SuperStore-Sale dataset. In addition to initally loading it, it also parses and analyses the dataset. Each attribute of the dataset gets classified in order to handle the attributes correctly later on. For the sake of convenience, four different classification types were used: numeric, date, nominal and unknown.

\item[PIXI.Container] \hfill \\
\ac{Pixi} features functionalities like container classes. \textit{PIXI.Container} is such a class and can be used to put objects in it, and scale or move those objects according to the base-canvas.

\item[AnimationQueue] \hfill \\
If an animation is assigned to particles, it is stored in the \textit{AnimationQueue}. This allows to create multiple canvas-based animations and process them sequentially.

\item[Canvas] \hfill \\
This class is one of the main classes inside the application. It is needed to not compare it with an HTML-Canvas object. It is responsible for creating the base-canvas of the web-application. Furthermore, it is used to change and update the visual appearance if any interaction happens. A key part of the canvas class is the render function shown in listing \ref{lst:canvas-render} on page \pageref{lst:canvas-render}. It starts off with calling the same function again as soon as possible. The caller-function \textit{requestAnimationFrame} ensures, that the next frame will only be shown if enough resources of the browser are available. Afterwards, all particles and visualisations are animated if something changed. The last line of the listing shows the \ac{Pixi}-renderer. Listing \ref{lst:canvas-autodecet-renderer} on page \pageref{lst:canvas-autodecet-renderer} shows its creation. It features the main canvas-size, background transparency and antialiasing. The renderer provides considerably more options which are not used.

%TC:ignore
\begin{lstlisting}[language=JavaScript, caption={Render function of the canvas class.}, label={lst:canvas-render}]
    render() {
      this.requestFrameID = requestAnimationFrame(this.render.bind(this));

      let areParticlesAnimating = this.particlesContainer.nextStep();
      let isNewVisualizationAnimating = this.visualization.nextStep();
      let isOldVisualizationAnimating = this.visualizationOld ? this.visualizationOld.nextStep() : false;

      if (!areParticlesAnimating && !isOldVisualizationAnimating &&
        !isNewVisualizationAnimating && this.animationQueue.length > 0) {

        this.animationQueue.pop()();
        this.particlesContainer.startAnimation();
        this.visualization.startAnimation();
        if (this.visualizationOld){
            this.visualizationOld.startAnimation();
        }
      }
      this.renderer.render(this.stage);
    }
\end{lstlisting}

\begin{lstlisting}[language=JavaScript, caption={Pixi's autodetect-renderer.}, label={lst:canvas-autodecet-renderer}]
    this.renderer = PIXI.autoDetectRenderer(this.width, this.height, {
        transparent: true,
        clearBeforeRender: true,
        antialias: true
    });
    document.body.appendChild(this.renderer.view);
\end{lstlisting}
%TC:endignore

\item[Visualization] \hfill \\
This class is the starting point and the parent class for all other types of visualisations. It stores a reference to the \textit{ParticleContainer} and provides functionality to move, scale or change visualisations.

\item[Overview] \hfill \\
Creating an instance of \textit{Overview} will show all data items as unit-based grid.

\item[D3] \hfill \\
This class is mainly used to wrap all functionality of the \ac{D3} library. It features functions like initialising the base-map as a \ac{SVG} and loading the needed TopoJSON data accordingly. \ac{D3} offers multiple projections for GeoJSON data. Listing \ref{lst:d3-map-init} on page \pageref{lst:d3-map-init} shows a part of the map initialisation with the projection used. The used projection is an extension of an Albers equal-area projection discussed in Chapter \ref{s:map-projections} on page \pageref{s:albers-equal-area-projection}. It is a United-States-centric composite projection. The lower forty-eight states of America are projected using the default Albers-equal-area projection. However, Alaska and Hawaii use a separate conic equal-area projection. The scale of Alaska is furthermore diminished. It is projected at $0.35\times$ its true relative area. The scale and translation of the map are set to use the available window space and center the map on the screen.

%TC:ignore
\begin{lstlisting}[language=JavaScript, caption={Map initialisation with a map projection, scale, and translation.}, label={lst:d3-map-init}]
    this.projection = this._d3.geo.albersUsa()
        .scale(width)
        .translate([this.width / 2, this.height / 2]);
\end{lstlisting}
%TC:endignore

Another key part of this class is featured in listing \ref{lst:d3-topojson} on page \pageref{lst:d3-topojson}. It shows how TopoJSON is used on the client. The filter function passed to the first \textit{TopoJSON.mesh}-function specifies that only internal state borders should be used. Thus, coastlines will not be drawn to retain detail around small islands and inlets. The filter function passed to the second \textit{TopoJSON.mesh}-function extends the first one by only showing each county boundary once. Thus, if two counties share the same border, it is only drawn once later on.

%TC:ignore
\begin{lstlisting}[language=JavaScript, caption={TopoJSON usage on the client with the adaption of merging all geographic information.}, label={lst:d3-topojson}]
    this.data.topojson.states = topojson.mesh(
        this.data.us,
        this.data.us.objects.states,
        (a, b) => {
            return a !== b;
        }
    );

    this.data.topojson.counties = topojson.mesh(
        this.data.us,
        this.data.us.objects.counties,
        (a, b) => {
            return (
                a !== b &&
                !(this.getCountyIdentifier(a) / 1000 ^ this.getCountyIdentifier(b) / 1000)
            );
        }
    );
\end{lstlisting}
%TC:endignore

Another important aspect of the \ac{D3} class is the \textit{calculateCentroids}-function shown in listing \ref{lst:d3-calculate-centroids} on page \pageref{lst:d3-calculate-centroids}. It pre-calculates a look-up dictionary for the given level of detail for each data item. Thus, it provides a huge performance boost when finding out the centroid of a unit later on.

%TC:ignore
\begin{lstlisting}[language=JavaScript, caption={Calculate a look-up dictionary for all data items depending on the level of detail.}, label={lst:d3-calculate-centroids}]
    calculateCentroids(levelOfDetail){
        const boundaries = this._topojson.feature(
            this.data.us,
            this.data.us.objects[levelOfDetail]
        ).features;

        this.centroids[levelOfDetail] = {};
        for(let boundary of boundaries){
            this.centroids[levelOfDetail][boundary.id] = this.path.centroid(boundary);
        }
    }
\end{lstlisting}
%TC:endignore

Furthermore, this class exposes two mandatory functions used to create aggregated thematic maps. Listing \ref{lst:d3-scales} on page \pageref{lst:d3-scales} features the scaling methods used in the application. The \textit{symbolScale}-method is based on a power scale with the input domain ranging from $0$ to $1000000$ and the output domain ranging from $0$ to $10$. The default exponent is $0.5$. If an input value exceeds the given domain, it is scaled and interpolated automatically respectively to the scale and a higher output range.
The colour scale is based on a quantise scale which are similar to linear scales, except they use a discrete rather than continuous range. The continuous input domain is divided into uniform segments based on the number of values in the output range. The output range is not shown in the listing because it depends on the chosen colour scheme. However, each scheme is represented by an array of 9 different colours. Thus, each range value can be expressed as a quantised linear function of the domain value.
Both scaling methods do not change their input domain throughout the application, even when changing the level of detail. The only change on a scale is the output colour of the \textit{colorScale}-method. As a result, the symbol size and colour used are not linked to the level of detail, but rather map the a particular input value always to the same output value.

%TC:ignore
\begin{lstlisting}[language=JavaScript, caption={Symbol-scale and colour-scale used for aggregated thematic maps.}, label={lst:d3-scales}]
    this.symbolScale = this._d3.scale.sqrt()
    .domain([0, 1e6])
    .range([0, 10]);

    this.colorScale = this._d3.scale.quantize()
    .domain([0, 1000]);
\end{lstlisting}
%TC:endignore

\item[BaseMap] \hfill \\
This is the second class inheriting from the \textit{Visualization} class. It features a single object of the \ac{D3} class. Therefore, it is possible that all deriving classes share the same object of \ac{D3} and thus, all its features and settings. Changing the level of detail of the \textit{BaseMap} will affect the map initialised in \textit{\ac{D3}} and hence, all deriving classes are using the same base-map again. This main purpose of this class is the Singleton object of \ac{D3} for all children.
The decision if and how the upcoming map should be animated or not handles each subclass on its own. All subclasses based on aggregation are implemented and animated with \ac{D3}. Therefore, the dot map is slightly different than these in the terms of structure and usage. Furthermore, all thematic maps based on aggregation are implemented with the method of using a static file containing all information. However, the aggregated values could also be calculated dynamically every time.

\item[DotMap] \hfill \\
The \textit{DotMap} class is the first one to show data on the base-map. It does not use the \ac{DOM} to create the dots (also called particles) on the map. It uses the already existing particles from the particle container inherited from the \textit{Visualization} class. Key part of this class is the initial position of all particles because the dot map is used as an overview of the used data. Listing \ref{lst:dot-draw-animated} on page \pageref{lst:dot-draw-animated} shows the initial placement. Each particle has geographical information denoted as \textit{particle.data.Longitude} and \textit{particle.data.Latitude} which is projected onto the map. The subsequent code saves the projection on the particle object and sets its size and position.

%TC:ignore
\begin{lstlisting}[language=JavaScript, caption={Particles on a dot map getting animated.}, label={lst:dot-draw-animated}]
    for(let particle of this.particles){
        point = [particle.data.Longitude, particle.data.Latitude];
        point = this.baseMap.projection(point);
        particle.coords = point;
        particle
        .setPosition(
            particle.coords[0]-(this.size/2),
            particle.coords[1]-(this.size/2)
        )
        .setSize(
            this.size,
            this.size
        );
    }

\end{lstlisting}
%TC:endignore

\item[ProportionalSymbolMap] \hfill \\
The implemented proportional symbol map is a multivariate classed proportional symbol map. The symbol size is scaled with the herein before mentioned symbol scale method. The population of the enumeration unit is the input domain of the method and therefore its symbol size. The second attribute is mapped with colour onto each symbol. The amount of orders of each enumeration unit is used. The amount is classified in a range of $9$ possible classes, where each class is represented by a colour of a sequential colour scale.
Drawing the multivariate classed proportional symbols is done by using \ac{D3}. First of all, it does not use the SuperStore-Sale file as a basis, and instead uses the preprocessed TopoJSON file. The class exposes functions like \textit{initSymbols}, \textit{colorSymbol}, \textit{scaleSymbols}, and so forth. Each function handles a single concern offering the best possible flexibility to create animated transitions to different kinds of visualisations.

\item[ChoroplethMap] \hfill \\
The implementation of this type of thematic map is a univariate classified choropleth map, where each coloured area represents the amount of orders.

\item[Cartogram] \hfill \\
The web-application implements a specific type of cartogram: a pseudo Demers cartogram. A true Demers cartogram would need links between adjacent features. However, the type of cartogram implemented tries to preserve locality instead of connectedness. Furthermore, it uses circles instead of squares. This allows to create an animated transition from proportional symbols to a cartogram by applying a force with some kind of gravity without the need to change the symbol.
Preserving locality is done by placing each circle as close as possible to its origin without overlapping. In order to deliver this kind of information to the user, a collision detection with gravity is used to animate the position of each circle. Listing \ref{lst:cartogram-part-tick} on page \pageref{lst:cartogram-part-tick} shows a small part of the force directed layout. The \textit{tick}-method can be compared to the before mentioned \textit{render}-method of the canvas class. It executes every frame if possible and applies collision detection and gravity to every symbol. \textit{gravity}-method is a key part is this concept. The attributes \textit{x0} and \textit{y0} represent the origin of each symbol whereas \textit{x} and \textit{y} are the current coordinates changed by the collision function. Thus, the gravity function ensures, that the symbols are as close to their location as possible.

%TC:ignore
\begin{lstlisting}[language=JavaScript, caption={A small part of the draw-function of the pseudo Demers Cartogramm-class}, label={lst:cartogram-part-tick}]
    tick(gravity) {
        this[this.id]
        .each(this.gravity(gravity))
        .each(this.collide(0.25))
        .attr("cx", d => { return d.x; })
        .attr("cy", d => { return d.y; });
    }

    gravity(k) {
        return d => {
            d.x += (d.x0 - d.x) * k;
            d.y += (d.y0 - d.y) * k;
        };
    }
\end{lstlisting}
%TC:endignore

\item[ParticlesContainer] \hfill \\
This container contains all particles and ensures, that they are animated in the desired order. Furthermore, it controls the speed of each animated particle.

\item[Particle] \hfill \\
A particle inherits the functionality of \textit{PIXI.Sprite} and contains information about its position, alpha value, and the data it represents. It exposes methods to draw a particle onto the map and to animate it.

\item[TransitionManager] \hfill \\
If the type of visualisation changes, an instance of this class gets called. It handles all animations concerning the transition. The exposed \textit{animate}-function needs two parameters as strings: the type of the current visualisation and the type of the upcoming one. With this information, it is possible to handle each transition separately. Section \ref{s:animated-transitions-implemented} on page \pageref{s:animated-transitions-implemented} discusses the implemented and animated transitoins in detail.
\end{description}

Another aspect of all thematic maps based on aggregation is the stored reference of the data shown (the allocation of \textit{this[this.id] = \ldots}). This reference is used to determine if the data on the map needs to be updated and redrawn or not.


\subsection{User study}
To answer the research question of this thesis, a user study yielding qualitative results was conducted. Such a study verifies that the implemented application fulfills its design goals.
The study setup was held in a laboratory environment with 14 participants. It was necessary that each participant does not know the purpose of proportional symbol maps, choropleth maps and cartograms. This requirement ensures that recognition of the animated transition between two visualisations is not combined with previous knowledge about the thematic map type.

The design of the conducted study was within-subject with think-aloud tasks and a post-task interview. Its primary objective was to check if the animated transition is able to point out strengths and weaknesses of each thematic map type. To achieve this, the task participants had to solve was of minor importance. It consisted of finding one of the areas in the United States with the most orders. Participants had to justify their answer with the chosen map type. Although, each implemented type of map would be sufficient to answer the question, another requirement was introduced. Participants had to look at every map type at least once. The justification of their answer lead to a post-task interview, where they were asked why they did or did not choose a specific type of map. The participants had to name the advantages and disadvantages of each thematic map if possible (as shown in chapter \ref{s:univariate-maps} on page \pageref{s:univariate-maps}). Hence, the results of the user study are in a binary format - either a participant could name a particular advantage or disadvantage or not.

The conducted user study was performed with 6 female and 8 male participants. They were between 21 and 47 years old. They all described themselves as technically affine with some experience in the domain of information visualisation. The residual chapter shows the results from the study.



