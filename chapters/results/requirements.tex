The practical application needs to meet some basic conditions, which are described below:

\begin{description}
\item[Scaling:] unit-based visualisations allow for showing a lot of items simultaneously and combined with animation, some patterns in data can be revealed. However, using a low amount of units will not result in showing patterns animations. Therefore, using a mid-sized dataset is a key part of the practical implementation. Defining a mid-sized dataset is nebulous and for this thesis, it is defined as a dataset consisting of 5000 to 15000 rows. This amount should be enough for an experiment to investigate the question of how an aggregation can be realized.

\item[Modularity:] the practical implementation should have a modular structure in order to keep it extendable. It should be possible to add further visualisations and transitions for more investigation later on.

\item[Client-side application:] in order to save costs and abstractions, a client-side approach is required. This allows to put the focus on the implementation of different thematic maps and an animated transition between them without worrying about latency of server response.

\end{description}