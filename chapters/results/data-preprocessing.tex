As the section before already showed, two different files are now accessible. However, in order to actually use these files, some preprocessing is needed. This chapter is divided into three parts. The first one will dicsuss the preparation of the Superstore-Sale dataset, the second one will explain how to get boundaries out of the downloaded zip-file and the third one will demonstrate how these two files can be merged.

\subsubsection{Preprocessing Superstore-Sale}
First of all, some numbers in this dataset are shown in a different format. Converting all numbers to the same format is essential.
Secondly, as already mentioned, the dataset only features attributes like country, city, state and postal code. In order to show the items on the map, it is necessary to expand each item with longitude and latitude. Listing \ref{lst:data-prep-latlong} on page \pageref{lst:data-prep-latlong} shows how the expansion can be accomplished when using NodeJs\footnote{See \href{https://nodejs.org/en/}{NodeJs} for more information.}. The listing uses Nominatim\footnote{See \href{https://nominatim.openstreetmap.org/}{Nominatim} for more information.} to decode geographical location strings to longitude and latitude. Furthermore listing \ref{lst:data-prep-latlong} shows, that each item is extended two more attributes: \textit{CountyId} and \textit{StateId}. These attributes denote the geographical id to the corresponding county and state name. As a starting point, a file containing all zip-codes with their corresponding ids was needed. Jgoodall provides such a file in his us-map repository\footnote{See \href{https://github.com/jgoodall/us-maps/}{his repository} for more information.}. Reading this file results in a lookup dictionary, where it is possible to search each postal code for its corresponding ids. However, Superstore-Sale does not provide a leading zero if a postal code is less than five characters, leading to a special case, which has to be considered.

%TC:ignore
\begin{lstlisting}[language=JavaScript, caption={Preprocessing Superstore-Sale with latitude and longitude}, label={lst:data-prep-latlong}]
let zipCodes = require("./zip-codes.json");
// Add latitude and longitude if there is a geo information
if (item.hasOwnProperty("Country") && item.hasOwnProperty("City") && item.hasOwnProperty("State")) {
    let url = `http://nominatim.openstreetmap.org/search?email=${config.email}&format=json&`;
    url += `country=${item.Country}&`;
    url += `state=${item.State}&`;
    url += `city=${item.City}`;

    const response = request("GET", url, {
        "headers": {
            "user-agent": "University of Applied Sciences Salzburg - Masterthesis Particles - MMT-M2014"
        }
    });
    const data = JSON.parse(response.getBody());
    item.Latitude = data[0].lat;
    item.Longitude = data[0].lon;
}

if (item.hasOwnProperty("Postal Code")) {
    let currentZip = item["Postal Code"];
    if(zipCodes.hasOwnProperty(currentZip)){
        let codes = zipCodes[currentZip];
        item.CountyId = codes.countyId[0];
        item.StateId = codes.stateId;
    } else {
        // superstore has no leading 0 of postal codes if code length < 5;
        currentZip = "0".concat(currentZip)
        if(zipCodes.hasOwnProperty(currentZip)){
            let codes = zipCodes[currentZip];
            item.CountyId = codes.countyId[0];
            item.StateId = codes.stateId;
        }
    }
}
\end{lstlisting}
%TC:endignore

\subsubsection{Preprocessing Cartographic Boundaries}
Section \ref{s:data-acquisition} on page \pageref{s:data-acquisition} already mentions how to automatically download and save the zip file containing cartographic boundaries of the United States. However, the zip-file by itself is not useful. The contents of the zip-file need to be converted first in order to actually being able to use the data. Listing \ref{lst:data-acqu-unzip} on page \pageref{lst:data-acqu-unzip} displays the make task for unzipping files. This task unzips \textit{gz\_2010\_us\_050\_00\_20m.zip} resulting in \textit{gz\_2010\_us\_050\_00\_20m.shp}. A shapefile is a common standard\footnote{See \href{https://doc.arcgis.com/en/arcgis-online/reference/shapefiles.htm}{the definition of a shapefile} for more information} for representing geospatial vector data.

%TC:ignore
\begin{lstlisting}[style={makefile}, caption={Make task for unzipping files}, label={lst:data-acqu-unzip}]
/*build/gz_2010_us_050_00_20m.shp*/: build/gz_2010_us_050_00_20m.zip
    unzip -od $(dir $@) $<
    touch $@
\end{lstlisting}
%TC:endignore

However, shapefiles cannot be used directly in today's browsers. Converting it to GeoJSON\footnote{See \href{http://geojson.org/}{GeoJSON} for more information.} first yields the result of having a usable file in the browser. GeoJSON is a format for encoding a variety of geographic data structures in a JSON structure. There are a variety of tools available converting shapefiles to GeoJSON. In order still automatize it, a command-line tool is needed and therefore TopoJSON\footnote{See \href{https://github.com/mbostock/topojson}{TopoJSON} for more information.} was used. TopoJSON is a simple extension to GeoJson that encodes topology. Listing \ref{lst:data-acqu-topo1} on page \pageref{lst:data-acqu-topo1} shows the usage of this command line tool. The option \textit{-o} simply denotes the input file, \textit{--} tells topojson that all options are passed in and \textit{counties} is the JSON-key all information is put in.

%TC:ignore
\begin{lstlisting}[style={makefile}, caption={Make task for converting shapefiles to geojson}, label={lst:data-acqu-topo1}]
/*data/counties.json*/: build/gz_2010_us_050_00_20m.shp
    node_modules/.bin/topojson \
        -o $@ \
        -- counties=$<
\end{lstlisting}
%TC:endignore

With this task, a file containing all cartographic boundaries for all counties exists. If this level of detail is too high, listing \ref{lst:data-acqu-topo2} and \ref{lst:data-acqu-topo3} on page \pageref{lst:data-acqu-topo2} and \pageref{lst:data-acqu-topo3} show how to create a file containing all state boundaries or only the main country boundaries. The \textit{--key='d.id.substring(d.id.search("S")+1, d.id.search("S")+3)'} is the aggregation base. Each item has an id which simply is the geo-id of the county. This geo-id is a combination of state-id and county-id and therefore using a substring of this id can be used to aggregate counties in the same state resulting in state boundaries.

%TC:ignore
\begin{lstlisting}[style={makefile}, caption={Make task for aggregating counties to states by state-id}, label={lst:data-acqu-topo2}]
/*data/states.json*/: data/counties.json
    ./../node_modules/.bin/topojson-merge \
        -o $@ \
        --in-object=counties \
        --out-object=states \
        --key='d.id.substring(d.id.search("S")+1, d.id.search("S")+3)' \
        -- $<
\end{lstlisting}

%TC:ignore
\begin{lstlisting}[style={makefile}, caption={Make task for converting state boundaries to a county boundary}, label={lst:data-acqu-topo3}]
/*data/us.json*/: data/states.json
    ./../node_modules/.bin/topojson-merge \
        -o $@ \
        --in-object=states \
        --out-object=country \
        -- $<
\end{lstlisting}
%TC:endignore

\subsubsection{Merging Superstore-Sale with Cartographic Boundaries}
This section mainly discusses pre-processing steps of one of the two methods mentioned to implement an aggregated thematic map (see chapter \ref{s:web-application} on page \pageref{s:web-application} for more information). The objective of merging Superstore-Sale with the already preprocessed boundary-files is to have one file including all aggregated values of orders per county or state. Such a file can be used to show all orders for each county or state without the need of dynamically calculating it.
In order to aggregate the orders on a county level of detail, listing \ref{lst:data-acqu-topo1} on page \pageref{lst:data-acqu-topo1} needs some adaptations with a new starting file. Using the already preprocessed Superstore-Sale dataset, it is possible to create a new file with the aggregated information. Listing \ref{lst:data-prep-amount} on page \pageref{lst:data-prep-amount} iterates the given dataset, creates a county-based dictionary counting all objects having the same county and writes the information back to a file.

%TC:ignore
\begin{lstlisting}[language=JavaScript, caption={Creating the file containing aggregation information}, label={lst:data-prep-amount}]
csv.fromPath("./path-to-preprocessed-superstore-sale.csv", {
    headers: true,
    delimiter: ";"
})
.on("data", function(item){

    let countyKey = `${item.StateId}${item.CountyId}`;
    countyDict[countyKey] = ++countyDict[countyKey] || 1;

})
.on("end", function(){

    let result = [];

    result.push([
        "GEO_ID",
        "ID",
        "AMOUNT"
    ]);

    for (let [key,value] of objectEntries(countyDict)) {
        result.push([
            `0500000US${key}`,
            `${key}`,
            `${value}`
        ]);
    }

    let ws = fs.createWriteStream(
        "./data/superstore-aggregated.csv",
        {encoding: "utf8"}
    );
    // comma seperation needed because of topojson
    csv.write(result, {headers: true, delimiter: ","}).pipe(ws);
});
\end{lstlisting}
%TC:endignore

This listing yields a file which can be used in combination with the already created topojson file. Listing \ref{lst:data-acqu-topo4} on page \pageref{lst:data-acqu-topo4} is an updated version of listing \ref{lst:data-acqu-topo1} on page \pageref{lst:data-acqu-topo1}. It extends the already mentioned task by combining the newly created aggregation file. The option \textit{--id-property} simply denotes the identification key for each item for both files. All \textit{--external-properties} needs a file as input parameter to look up \textit{--properties} which cannot be found in the original input file from the task. Thus, all \textit{--properties} listed in the task are available as a key-value pair later on.

%TC:ignore
\begin{lstlisting}[style={makefile}, caption={Make task for merging a file with topojson}, label={lst:data-acqu-topo4}]
/*data/counties.json*/: build/gz_2010_us_050_00_20m.shp
    ./../node_modules/.bin/topojson \
        -o $@ \
        --id-property 'GEO_ID' \
        --external-properties ./data/superstore-aggregated.csv \
        --properties 'geoId=GEO_ID' \
        --properties 'stateId=STATE' \
        --properties 'countyId=COUNTY' \
        --properties 'county=NAME' \
        --properties 'orders=AMOUNT' \
        -- counties=$<
\end{lstlisting}
%TC:endignore
