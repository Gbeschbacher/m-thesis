Tableau uses a very interesting sample dataset to show basic features. It is called "Superstore-Sale" and contains approximately 10000 rows. Superstore-Sale features a lot of different attributes for each item. It comes close to the example of e-commerce mentioned in chapter \ref{s:geovis-practical} on page \pageref{s:geovis-practical}. The dataset mainly consists of information about office products, which have been bought in the period of time from 2011 to 2014. Furthermore each order in the dataset includes some parts of the delivery address. This information about country, city, state and postal code allows to use the dataset with a map because it features some kind of geographical information. However, before being able to use the dataset on a map, it must be expanded by longitude and latitude first. This part is explained in detail in section \ref{s:data-preprocessing} on page \pageref{s:data-preprocessing}.

Having a dataset with interesting information leads to the second part of this section. A thematic map does not only consist of its symbols or coloured areas, it also needs to show the geographical conditions in form of a map. "Superstore-Sale" only includes items from the United States, thus a map of this location is needed. The U.S. Census Bureau publishes cartographic boundaries as shapefiles for thematic mapping. They provide multiple resolutions for different use cases. For the chosen dataset, the lowest resolution will show enough features in the map. Being able to see forests, streets, and so forth, in the map is not needed to show an overview of a commercial dataset. An advantage using a low resolution dataset is the file size. This must be considered, because we will put all information to the client, thus delivering 1 \ac{mb} or 100\ac{mb} of information is significant in terms of loading time of the web application.
Furthermore, the topicality of the cartographic boundaries needs to be considered. Usually county boundaries do not change frequently, so it is acceptable to use the decennial census rather than the most recent release. Listing \ref{lst:data-acqu-zip} on page \pageref{lst:data-acqu-zip} shows an automated way of downloading the decennial version of the lowest resolution cartographic boundaries. It is accomplished with GNU-Make\footnote{See \href{https://www.gnu.org/software/make/}{GNU Make} for more information.}. The target name without the directory, \textit{gz\_2010\_us\_050\_00\_20m}, implicitly has some information:

%TC:ignore
\begin{lstlisting}[style={makefile}, caption={Make task for downloading cartographic boundaries}, label={lst:data-acqu-zip}]
/*build/gz_2010_us_050_00_20m.zip*/:
    mkdir -p $(dir $@)
    curl -o $@ http://www2.census.gov/geo/tiger/GENZ2010/$(notdir $@)
\end{lstlisting}
%TC:endignore

\begin{itemize}
\item \textbf{2010} is the release year of the file
\item \textbf{us\_050\_00} refers to boundaries of the united states (two characters abbreviation and the country id)
\item \textbf{20m} denotes the resolution (1:20.000.000)
\end{itemize}

The mentioned make task will download the zip-file from the census bureau and save it in the build directory.

