To answer the research question of this thesis, a user study yielding qualitative results was conducted. Such a study verifies that the implemented application fulfills its design goals.
The study setup was held in a laboratory environment with 14 participants. It was necessary that each participant does not know the purpose of proportional symbol maps, choropleth maps and cartograms. This requirement ensures that recognition of the animated transition between two visualisations is not combined with previous knowledge about the thematic map type.

The design of the conducted study was within-subject with think-aloud tasks and a post-task interview. Its primary objective was to check if the animated transition is able to point out strengths and weaknesses of each thematic map type. To achieve this, the task participants had to solve was of minor importance. It consisted of finding one of the areas in the United States with the most orders. Participants had to justify their answer with the chosen map type. Although, each implemented type of map would be sufficient to answer the question, another requirement was introduced. Participants had to look at every map type at least once. The justification of their answer lead to a post-task interview, where they were asked why they did or did not choose a specific type of map. The participants had to name the advantages and disadvantages of each thematic map if possible (as shown in chapter \ref{s:univariate-maps} on page \pageref{s:univariate-maps}). Hence, the results of the user study are in a binary format - either a participant could name a particular advantage or disadvantage or not.

The conducted user study was performed with 6 female and 8 male participants. They were between 21 and 47 years old. They all described themselves as technically affine with some experience in the domain of information visualisation. The residual chapter shows the results from the study.


