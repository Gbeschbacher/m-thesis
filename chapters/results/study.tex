To answer the research question of this thesis, a user study yielding qualitative results was conducted. Such a study verifies that the implemented application fulfills its design goals.
The study setup was held in a laboratory environment with 14 participants. It was necessary that each participant does not know the purpose of proportional symbol maps, choropleth maps and cartograms. This requirement ensures that recognition of the animated transition between two visualisations is not combined with previous knowledge about the thematic map type.

The design of the conducted study was within-subject with think-aloud tasks and a post-task interview. Its primary objective was to check if participants understand the relationship between different geo-spatial visualisations and also help them to extract strengths and weaknesses of each visualisation type. To achieve this, the task participants had to solve was of minor importance. It consisted of finding one of the areas in the United States with the most orders normalized by population. Participants had to justify their answer with the chosen map type. Although, each implemented type of map would be sufficient to answer the question, another requirement was introduced. Participants had to look at every map type at least once. The justification of their answer lead to a post-task interview, where they were asked why they did or did not choose a specific type of map. The participants had to name the advantages and disadvantages of each thematic map if possible (as shown in Chapter \ref{s:univariate-maps} on page \pageref{s:univariate-maps}).
To conclude the study, they had to rate the usefulness of the transitions on a scale from 1 to 5 where 1 denotes the transition as unhelpful and 5 as helpful. Hence, the results of the user study are in a binary format - either a participant could name a particular advantage or disadvantage or not and include a score for the transition overall.

The conducted user study was performed with six female and eight male participants. They were between 21 and 47 years old. They all described themselves as technically affine with some experience in the domain of information visualisation and no experience with transitions between visualisations. The residual Chapter shows the results from the study.

Revealing a general pattern of a distribution is the main objective of a dot map. 10 out of 14 participants could name the purpose of this thematic map. Even more participants (11 out of 14) said that it is not possible to determine exact quantities in a dot map due to underestimation of high density areas.

The major drawback of a proportional symbol map was detected by 10 out of 14 attendees. According to their feedback, the east coast of the United States was very hard to read due to close locations with high population and therefore overlapping symbols. However, only six people found that the estimation of quantities is less tedious. Seven people were able to name the advantage of a non-existing areal bias.

Finding hot spots was identified by 100\% of the participants as the primary goal of a choropleth map. Furthermore, nine attendees could also name the areal bias as a disadvantage of a choropleth map.

The distortions a cartogram comes along with are identified by 12 attendees as the main weakness of this type of map. Nonetheless, three participants discovered that cartograms emphasize the attribute mapped to the size of the symbol and eliminate visual biases due to their abstractness.

In addition, 10 participants rated the transition between the thematic maps with a score between three and five. The mean score of the transition is 3,21.