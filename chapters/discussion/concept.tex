% marc fragen, was hier noch stehen soll
\subsubsection{Analysis-Framework}
The implemented web application is based on two datasets: a tabular dataset (SuperStore-Sale) and a geo-spatial one (boundaries). These datasets are statically preprocessed and used. This information already answers the question of what the application visualises.

\cbstart
Furthermore, the main objective of the visualisations is to consume information, according to Figure \ref{fig:why} on page \pageref{fig:why}. In addition, users of the application should discover previously unkown knowledge. Considering the performed user study and its task, a user should discover an area in the United States which has the most orders normalized by population.
\cbend

Overall, the application uses static encodings like colour, size and shapes for the different types of maps and motion when changing the visual appearance. This already leads to the possibilities in manipulation. Changing the level of detail, colour scheme and particle speed allow for exploring different settings. However, reducing or changing the displayed information is not possible at all. The implemented types of maps have very specific characteristics, making a consistent aggregation or filtering of information impracticable.
