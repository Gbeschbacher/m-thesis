The study results suggest that the transition mostly supported pointing out the advantages and disadvantages of geo-spatial map visualisations. The drawbacks of each type of visualisation were found by 64\%. Furthermore, the strengths of a dot map and a choropleth map were recognised by 71\%. The most widespread types of thematic maps are dot and choropleth maps. This could indicate a correlation with the slightly better results of those two types. The proportional symbol map and the Pseudo-Demers cartogram had less positive results. 50\% of the participants could point out the advantages of a proportional symbol map and only 21\% those of a cartogram.

The score indicating the usefulness of the transition overall is relatively low compared to the mostly positive results of the binary questions. However, the mean score of the transition (3,21) indicates, that the transition was received slightly positive. Thanks to the feedback the participants provided, it is possible to say that more animation interaction and different types of transitions would significantly affect the score in a positive way. Some suggested implementing play and pause features to keep track of particular stops, whereas other ones asked for more variety in transitions. Therefore, the feedback can be used for future work which could lead to a higher score for the transition.

However, the study results also go in line with the statements made by \citeauthor{Munzner2014}. She said using motion as a visual encoding channel is an effective one but hard to master \iacite{Munzner2014}. Every person has different perception of motion, e.g. the speed of moving objects. This was also observable in the conducted study. Participants with more experience in the domain of visualisation used a higher particle speed compared to inexperienced ones.

Thus, relating the primary research question of this thesis with the study results suggest that transitions do help in pointing out strengths and weaknesses of thematic maps. The generalised research question of which tasks can be supported by particle aggregation in a thematic map would need further investigation with different use cases in order to answer it properly.