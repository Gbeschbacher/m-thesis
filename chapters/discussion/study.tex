The study results showed that the transition mostly supported pointing out the advantages and disadvantages of the thematic maps. The drawbacks of each type of visualisation were found by at least 64\%. Furthermore, the strengths of a dot map and a choropleth map were found at least 71\%.
The most widespread types of thematic maps are dot and choropleth maps. This could indicate a correlation with the slightly better results of those two types.
Therefore, the rarity of cartograms could also be a reason why only 21\% could point out its strengths. However, this stands in contrast with 85\% of the participants knowing the drawbacks of such a visualisation after the user study.
The same fact could be the reason  why only a maximum of 50\% of the attendees could name the advantages a proportional symbol map.

The score indicating the usefulness of the transition overall is relatively low compared to the mostly positive results of the binary questions. However, 55,25\% (the mean score of the transition) also indicates, that the transition was received slightly positive. Thanks to the feedback the participants provided, it is possible to say that the lack of animation interaction and transition variety were the reason why the score ended up so low. Some suggested implementing play and pause features to keep track of particular stops, whereas other ones asked for more variety in transitions. Therefore, the feedback can be used for future work which could lead to a higher score for the transition.

However, the study results also go in line with the statements made by \citeauthor{Munzner2014}. She said using motion as a visual encoding channel is a very effective one but very hard to master \iacite{Munzner2014}. Every person has different perception of motion, e.g. the speed of moving objects. This was also observable in the conducted study. Participants with more experience in the domain of information visualisation used a higher particle speed compared to inexperienced ones.

Thus, relating the primary research question of this thesis with the study results show that transitions do help in pointing out strengths and weaknesses of thematic maps. The generalised research question of which tasks can be supported by particle aggregation in a thematic map would need further investigation with different use cases in order to answer it properly.