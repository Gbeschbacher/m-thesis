%add more about generating data and data overall
There has never been such a quantity of data available in history and the global amount of information has never grown so strong and continuous. Every day 2.5 Exabytes of new data is generated and by 2020 an annual growth of 35 Zettabytes is predicted \iacite{IBM.2013}. Most scientific experiments and simulations of different types like chemistry, medicine or physics generate more and more data. The "Large Hadron Collider" of CERN in Genf produced 13 petabytes of data in 2010 \iacite{Brumfield.2011}. Economy, especially merchandise trade generate data in large measures. Each and every one of us generates data every day. Partly, this happens knowingly with activities in social media \textendash\ facebook alone produces 500 terabytes every day \iacite{Tam.2012} \textendash\ but unconsciously we constantly produce data with various sensors in our smartphones.

%add more about information overload and turning information into knowledge
Processing and analysis of data gets increased economic and scientific interest. Searching for useful and usable information is commonly known as "Data Mining".

Despite recent technical advantages of search engines and content provider services the information overload problem still remains a crucial issue for many application fields. Finding the right piece of information in huge information spaces is a tedious and time consuming task. Visualisations have shown to be an effective way to deal with the overload issue providing the possibility to display and explore a huge set of data points at the same time. However, creating useful visual representations of data typically requires domain knowledge \iacite{Nazemi.2013}.

% Assume a dataset including retail information.
Assuming a visualisation about e-commerce data exists, there is often the problem that it is not intuitive and easy to understand. If a shop wants to find out where to open their next warehouse in order to provide a better delivery service, it is needed to find out where most of the orders come from. For this type of task, multiple types of visualisations exist. However, some may not be as appropriate as others. Giving the user control to change the visual appearance may be enough. Nonetheless, animating the transition between visual appearances could result in a better readability of each visualisation. Considering the aforementioned task, any kind of thematic map would be an appropriate type of visual representation.

Thus the importance of the field of information visualisation grows with the amount of generated data. Information visualisation utilises the human capability of understanding pictures better than just lists of numbers. It allows the discovery of knowledge in data that would not be possible with other means.